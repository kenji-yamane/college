\documentclass{beamer}
\usetheme{PaloAlto}
\usepackage{graphicx, color, hyperref}
\usepackage[utf8]{inputenc}
\usepackage[T1]{fontenc}
\usepackage{listings}

\title[Apresentação]{Aula 05 - Bizus e Guloso}
\author[\textit{Kenji Yamane}]{Kenji Yamane}
\date{Março de 2020}

\setlength{\parindent}{20 pt}

\begin{document}

\begin{frame}

\titlepage

%\begin{figure}
%	\includegraphics[width=30 pt]{c-symbol}
%	\hspace{20 pt}
%	\includegraphics[width=30 pt]{cpp-symbol}
%	\label{c and cpp}
%\end{figure}

\end{frame}

\begin{frame}

\frametitle{Tópicos da aula}

\tableofcontents

\end{frame}

\section{Mais bizus}
	
	\begin{frame}
	\frametitle{EOF em C++}
		Na aula passada foi mostrado como fazer EOF com scanf.
		Aproveita-se para adicionar que caso você queira simular um
		"fim de arquivo" para verificar se o seu programa realmente
		para, pode-se apertar ctrl D no linux, e se for no windows basta
		apertar ctrl Z. Agora, como que se faz a mesma coisa com cin?
		É até mais simples, como a maioria das coisas em C++ são.
		Enquanto scanf retorna -1 ao não conseguir mais ler algo, cin
		retorna NULL (ponteiro que aponta para nada), que
		efetivamente funciona como uma booleana de falso. Portanto,
		em código (o endl é tipo um \textbackslash n):
		\begin{block}{Código 1}
		\hspace{10 pt} while(cin >> N)\\
		\hspace{20 pt} cout << N << endl;
		\end{block}
	\end{frame}

	\begin{frame}
	\frametitle{Memset}
		Da biblioteca string.h, memset é uma função bem simples
		mas que é útil também, por ser prática. Caso você queira zerar
		todos os valores de uma matriz memo, por exemplo:
		\begin{block}{Código 2}
		\hspace{10 pt} memset(memo, 0, sizeof(memo));
		\end{block}
		De fato já é possível fazer isso com listas inicializadoras de
		C com menos código até, porém memset pode servir para
		inicializar todas as casas para -1 caso você queira:
		\begin{block}{Código 3}
		\hspace{10 pt} memset(memo, -1, sizeof(memo));
		\end{block}
		Isso vai ser bem útil quando visitarmos programação
		dinâmica posteriormente.

		Obs: usem memset somente com 0 e -1 (estranho eu sei).
	\end{frame}

	\begin{frame}
	\frametitle{Estruturas como argumento de funções}
		Muitos devem lembrar de variáveis de escopo das aulas de
		C. Quando você passa uma variável para uma função, essa
		variável não é efetivamente passada, mas sim é criada uma
		cópia dela. A mesma coisa acontece caso você queira passar
		uma estrutura da STL. Porém, é possível passar a estrutura,
		sem ser somente uma cópia, para caso você queira fazer
		alterações que sejam observadas na main também:
		\begin{block}{Código 4}
		\hspace{10 pt} \textcolor{teal}{// passando c\'opia}\\
		\hspace{10 pt} int func(int aux, vector<pair<int, int> > v);\\
		\hspace{10 pt} \textcolor{teal}{// passando o vetor}\\
		\hspace{10 pt} int func(int aux, vector<pair<int, int> > \&v);
		\end{block}
	\end{frame}

	\begin{frame}
	\frametitle{Sort - inversão}
		Lembram do sort? Uma das razões dele ser tão bizu é que
		ele ordena em O(NlogN), o mais rápido que tem para um
		algoritmo de ordenação em que o espaço amostral é genérico.
		Por padrão ele deixa na ordem crescente, mas é possível ordenar
		em ordem decrescente de uma forma parecida com quando se
		inverte a priority\_queue:
		\begin{block}{Código 5}
		\hspace{10 pt} sort(v, v + N, greater<int>());
		\end{block}
	\end{frame}

	\begin{frame}
	\frametitle{Sort - em outras coisas}
		Vocês podem ordenar outras estruturas mais complicadas
		também, basta dar ao sort a forma de comparação que você
		quer basear sua ordenação. Por exemplo se você quer ordenar
		um vetor de pairs em ordem crescente de primeiro do par,
		primeiro crie sua função comparativa:
		\begin{block}{Código 6}
		\hspace{10 pt} bool comparePairs(pair<int, int> a, pair<int, int> b)\\
		\hspace{20 pt} return (a.first < b.first);
		\end{block}
		E então insira no sort, (supondo v ser um vector de pairs):
		\begin{block}{Código 7}
		\hspace{10 pt} sort(v.begin(), v.end(), comparePairs);
		\end{block}
	\end{frame}

	\begin{frame}
	\frametitle{Last touches}
		O seguinte código é uma feature que tem nas últimas
		versões de C++ (talvez seu compilador não tenha), é de
		varredura e funciona com deque, vector, set, map...
		\begin{block}{Código 8}
		\hspace{10 pt} for (auto c : mapa)
		\end{block}
		Escrever and e or ao invés de \&\& e ||.
		\begin{block}{Código 9}
		\hspace{10 pt} if (aux1 == 8 and aux2 == 9 or aux3 < 10)
		\end{block}
		E, pode-se alterar o tipo de uma variável, se chama cast:
		\begin{block}{Código 10}
		\hspace{10 pt} double var = 9.8;\\
		\hspace{10 pt} printf("\%d\textbackslash n", (int)var);
		\end{block}
	\end{frame}

	\section{Guloso}

	\begin{frame}
	\frametitle{Assunto do dia}
		Agora sim falemos do foco da aula de hoje! O segundo
		paradigma do nosso planejamento, guloso.
		\begin{itemize}
			\item Força Bruta
			\item \textbf{Guloso}
			\item Dividir para Conquistar
			\item Programação Dinâmica
		\end{itemize}
	\end{frame}

	\begin{frame}
	\frametitle{Como funciona o guloso}
		Considere uma resposta ótima outra forma de dizer
		resposta certa. Um algoritmo guloso é aquele que sempre
		escolhe localmente a opção ótima, com a esperança de no final
		alcançar a resposta ótima global. Por exemplo, considere o
		seguinte problema: dado um conjunto de n moedas cada uma
		com um valor diferente, e um determinado valor V, qual o
		número mínimo de moedas para que os valores delas somem V?
		Por exemplo, se n = 4 e moedas = \{25, 10, 5, 1\} e V = 22,
		podemos expressar V como \{1, 1, 5, 5, 10\} ou como \{5, 5, 5,
		5, 1, 1\}, porém é requisitado o menor número possível de
		moedas! Perceba que é possível tentar algum algoritmo de força
		bruta, porém como normalmente acontece ele decerto seria
		muito lento (mais que O(2 N )).
	\end{frame}

	\begin{frame}
	\frametitle{Problema da moeda}
		Um algoritmo guloso seria sempre pegar a maior moeda
		que não ultrapasse o valor V. O problema localmente é escolher
		alguma moeda, e a escolha localmente ótima seria escolher a
		maior moeda possível com a esperança de, no final, ter o
		conjunto com o menor número possivel de moedas. Faz sentido,
		não? No caso de moedas = \{25, 10, 5, 1\} e V = 42, esse
		algoritmo pegaria primeiro 25, e faltaria 42 - 25 = 17 para
		completar V. Depois, 10, faltando então 17 - 10 = 7, depois 5,
		1, e finalmente 1. O conjunto seria \{25, 10, 5, 1, 1\}, e a
		resposta 6, que de fato está certa. O código correspondente
		encontra-se no próximo slide.
	\end{frame}

	\begin{frame}[fragile]
	\frametitle{Código 11}
		\begin{lstlisting}
...
    int n, V, coins[10000], ans = 0, i;
    cin >> n
    for (i = 0; i < n; i++) cin >> coins[i];
    cin >> V;

    sort(coins, coins + n, greater<int>());
    i = 0;
    while (V > 0)
        if (coins[i] > V) i++;
        else V -= coins[i], ans++;
    cout << ans << endl;
    return 0;
}
		\end{lstlisting}
	\end{frame}

	\begin{frame}
	\frametitle{Problema da moeda}
		Bem rápido! Algo em torno de O(NlogN). Porém
		\textbf{receberia wrong answer}. A resposta para esse problema não é
		um algoritmo guloso. Considere um outro caso, coins = \{4, 3,
		1\} e V = 6. Por guloso, a solução seria \{4, 1, 1\}, mas outra
		possibilidade é \{3, 3\}, menor ainda. O que quer dizer que
		procurar pelo ótimo local não irá te levar ao ótimo global, nesse
		caso! Porém há problemas em que isso vale. Diz-se que eles
		tem a \textbf{propriedade gulosa}, algo que vocês verão que é bem
		difícil de se confirmar se de fato tem. Pode-se concluir assim
		que esse paradigma tende a levar \textit{wrong answer}, porém evita
		bastante \textit{time limit exceeded}. A solução de fato é por
		programação dinâmica, o último paradigma a ser visto.
	\end{frame}

	\begin{frame}
	\frametitle{Comprando gasolina}
		Um outro problema (dessa vez em que o guloso funciona):
		suponha que existam n distribuidoras de gasolina, cada uma
		vendendo a um preço $p_i$ ao litro e com $e_i$ litros no estoque. Se
		você precisa de V litros de gasolina, qual a menor quantia
		possível que você gastaria? Considerar o problema local, que
		seria a escolha da distribuidora, e escolher a opção ótima que é
		a que vender mais barato funciona nesse caso! O problema
		possui a propriedade gulosa. O código está amostrado a seguir.
	\end{frame}

	\begin{frame}[fragile]
	\frametitle{Código 12}
		\begin{lstlisting}
...
    int n, V, ans = 0;
    pair<int, int> gas[10000]; //pi, ei
    cin >> n
    for (i = 0; i < n; i++)
        cin >> gas[i].first >> gas[i].second;
    cin >> V;
    sort(gas, gas + n, comparePair);
    for (int i = 0; V > 0; i++)
        int aux = min(V, gas[i].second,
        ans += aux*gas[i].first,
        V -= min(V, gas[i].second);
    cout << ans << endl;
    return 0;
}
		\end{lstlisting}
	\end{frame}

	\begin{frame}
	\frametitle{Cobrindo o intervalo}
		Normalmente nos códigos de algoritmos gulosos aparece a
		ordenação dos dados, para poder ordená-los de acordo com a
		busca local pelo ótimo que você quer fazer. Em outro problema,
		dados N segmentos de linha (no eixo x), e suas abscissas de
		início e fim [$ini_i$] , [$fim_i$], qual a quantidade mínima de segmentos
		que você precisa para cobrir um segmento [0, S]? O problema
		localmente é qual segmento você escolhe, e gulosamente se
		escolhe sempre o segmento com menor abscissa de início. Se as
		abscissas de início forem iguais, escolhe-se gulosamente a que
		tiver maior abscissa de fim. Porém, se a absicssa de fim for
		menor do que eu cobri até agora, eu a ignoro, e se a abscissa de
		início for maior do que eu cobri até agora, é impossível cobrir. É
		outro problema com a propriedade gulosa.
	\end{frame}

	\begin{frame}[fragile]
	\frametitle{Código 13}
		\begin{lstlisting}
#include <bits/stdc++.h>
#define pii pair<int, int>

using namespace std;

int N, S, reach, ans;
// reach eh ate onde foi coberto, comeca no 0
pii seg[10000]; //ini, fim

bool comparePair(pii a, pii b) {
    if (a.first == b.first)
        return a.second > b.second;
    else
        return a.first < b.first;
}
		\end{lstlisting}
	\end{frame}

	\begin{frame}[fragile]
	\frametitle{Código 13}
		\begin{lstlisting}
int main() {
    cin >> N;
    for (int i = 0; i < N; i++)
        cin >> seg[i].first >> seg[i].second;
    cin >> S;
    sort(seg, seg + N, comparePair);
    for (int i = 0; i < N; i++)
        if (seg[i].first <= reach) {
            if (seg[i].second > reach)
                reach = seg[i].second, ans++;
        }
        else cout << "unable to cover" << endl;
    cout << ans << endl;
    return 0;
}
		\end{lstlisting}
	\end{frame}

	\begin{frame}
	\frametitle{Apagando e Ganhando}
		Comentando sobre o problema do contest de vocês,
		apagando e ganhando, alguns devem ter sido tentados por uma
		ideia gulosa que na realidade dava wrong answer. Relembrando
		o problema, dado um determinado número, e uma quantidade
		de dígitos a serem deletados, determine o maior número que
		pode ser gerado nessa brincadeira. Como alguns perceberam,
		sempre buscar deletar gulosamente os menores dígitos não
		funciona. Porém um algoritmo accepted envolve um príncipio
		guloso: sempre deletar de modo a trazer os maiores dígitos para
		esquerda.
	\end{frame}

	\section{Problemas para treinar!}

	\begin{frame}
	\frametitle{Problemas de guloso}
	\begin{itemize}
	\item \textcolor{blue}{\underline{\href{https://onlinejudge.org/index.php?option=com_onlinejudge&Itemid=8&page=show_problem&problem=1597}{Maximum Sum (II)}}}
	\item \textcolor{blue}{\underline{\href{https://www.urionlinejudge.com.br/judge/pt/problems/view/1643}{Converter Quilômetros para Milhas}}}
	\item \textcolor{blue}{\underline{\href{https://www.urionlinejudge.com.br/judge/pt/problems/view/2095}{Guerra}}}
	\item \textcolor{blue}{\underline{\href{https://www.urionlinejudge.com.br/judge/pt/problems/view/2115}{Produção em Ecaterimburgo}}}
	\item \textcolor{blue}{\underline{\href{https://www.urionlinejudge.com.br/judge/pt/problems/view/1661}{Comércio de Vinhos na Gergóvia}}}
	\end{itemize}
	\end{frame}

\end{document}
