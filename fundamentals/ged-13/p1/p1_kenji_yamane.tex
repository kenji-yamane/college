\documentclass{article}[twocolumn]
\usepackage[pdftex]{graphicx}
\usepackage[utf8]{inputenc}
\usepackage[brazil]{babel}
\usepackage{subfigure}
\usepackage{mathtools}
\usepackage{amsmath}
\usepackage{float}

\title{RESOLUÇÃO DE PROVA: GED-13}
\author{Kenji Yamane}

\begin{document}
	\begin{figure}[H]
		\centering
		\includegraphics[width=10cm]{ita_logo.png}
	\end{figure}
	\begin{center}
		\textbf{\normalsize{INSTITUTO TECNOLÓGICO DE AERONÁUTICA}}\\
		\vspace{2.5cm}
		\large{Kenji de Souza Yamane}\\
		\vspace{4.7cm}
		\textbf{RESOLUÇÃO DA PROVA}\\
		\large{GED 13 - PROBABILIDADE E ESTATÍSTICA}\\
		\vspace{4.7cm}
		\small{São José dos Campos – SP\\2020}
	\end{center}
	\newpage
	\tableofcontents
	\newpage
	\section{Introdução}
	O presente trabalho contém a solução da Prova I da matéria GED 13 - Probabilidade e
	Estatística do segundo ano do fundamental do ano de 2020.

	As resoluções foram efetuadas, em sua maioria, a partir da linguagem de programação R, tendo
	como plataforma o \textit{Linux}, sobre linha de comando.

	As linhas de código foram diretamente copiadas da linha de comando do sistema operacional
	e inseridas em um código de \LaTeX, em ambiente \textit{verbatim}, sem nenhuma alteração.
	O código em \LaTeX, gerou então este arquivo.
	
	Como foram diretamente retirados da interface do \textit{Linux}, \textgreater, é indicativo
	de comando, e + é indicativo de comando sequenciado.
	\newpage
	\section{Regra de \textit{Bayes}}
	A fibrose cística, uma doença pulmonar fatal, é uma das doenças autossômicas recessivas mais
	comuns. Isso é verdade em parte porque existem mais de 1500 formas mutacionais do gene
	regulador da transmembrana da fibrose cística (CFTR) (Ratbi et al.
	2007)\footnote{http://humrep.oxfordjournals.org/cgi/content/full/22/5/1285.}. Com os métodos
	de diagnóstico atuais, a taxa de verdadeiro positivo (isto é, sensibilidade) para detectar
	mutações CFTR é de aproximadamente 0,99; portanto, a taxa de falso negativo é 0,01.
	A taxa de verdadeiro negativo (ou seja, especificidade) deste teste é geralmente assumida
	como 100\%, mas ignora o erro humano. Como resultado, vamos deixar essa probabilidade ser
	0,998 e deixar a probabilidade de falso negativo ser 0,002. Esses dados estão resumidos na
	Tabela \ref{tab:diagnostic} mostrada abaixo.\\
	Taxa de verdadeiro-positivo = P(POS $|$ CF+) = 0,99\\
	Taxa de falso-positivo = P(POS $|$ CF-) = 0,002\\
	Taxa de verdadeiro-negativo = P(NEG $|$ CF-) = 0,998\\
	Taxa de falso-negativo = P(NEG $|$ CF+) = 0,01\\
	Uma questão de grande interesse é “qual a probabilidade de uma pessoa ser portadora de alguma
	variedade  mutacional  de  CFTR,  mas  não  saber  disso  porque  seu  teste  foi negativo?”.
	Ou  seja,  "qual  o  valor  da P(CF+  $|$  NEG)  ?".  Não  sabemos  isso,  embora tenhamos uma
	estimativa para a probabilidade do resultado condicional, P(NEG $|$ CF+) = 0,01. Haverá dois
	estados possíveis da natureza resultando nas \textit{a priori}\footnote{Pelo que consultei,
	a locução latina a priori fica invariável quando colocada no plural.}: P(CF+) e P(CF-).
	Como resultado, configuramos nosso problema específico da seguinte maneira:
	\begin{equation}
		P(CF+|NEG) = \frac{P(CF+)P(NEG|CF+)}{\sum_{k=1}^{2}P(CF+_{k})P(NEG|CF+_{k})}
	\end{equation}
	\begin{table}[H]
		\centering
		\begin{tabular}{ccc}
			\hline
			Diagnóstico & CF+ & CF-\\
			\hline
			POS & 0.99 & 0.002\\
			NEG & 0.01 & 0.998\\
			\hline
		\end{tabular}
		\caption{Detecção de mutações CFTR.}
		\label{tab:diagnostic}
	\end{table}
	Nossas a priori referem-se as probabilidades de presença / ausência de mutações CFTR na
	população em geral. Essas probabilidades, por sua vez, dependem da etnia
	(Tabela-\ref{tab:priori}).
	\begin{table}[H]
		\centering
		\begin{tabular}{cc}
			\hline
			Etnia & Probabilidade de ser portador\\
			\hline
			Caucasiano Não Hispânico & 0.040\\
			Hispano-americano & 0.017\\
			Afro-americano & 0.016\\
			Judeu Ashkenazi & 0.042\\
			Asiático-americano & 0.011\\
			\hline
		\end{tabular}
		\caption{Risco de ser portador de FC para cinco grupos étnicos.}
		\label{tab:priori}
	\end{table}
	Determine os itens a seguir.\\
	Criou-se a seguinte função em R:
	\begin{verbatim}
		> calculate <- function(p_cf_mais) {
		+     p_neg_cf_mais <- 0.01
		+     p_neg_cf_menos <- 0.998
		+     p_cf_menos <- 1 - p_cf_mais
		+     p_neg_inter_cf_mais <- p_cf_mais*p_neg_cf_mais
		+     p_neg <- p_cf_mais*p_neg_cf_mais + p_cf_menos*p_neg_cf_menos
		+     return (p_neg_inter_cf_mais/p_neg)
		+ }
	\end{verbatim}
	\subsection{a}
	A  probabilidade  P(CF+  $|$  NEG)  para  Caucasianos  Não  Hispânicos.  Esse  resultado
	representa aproximadamente quantos indivíduos por milhão de habitantes?
	\begin{verbatim}
		> calculate(0.040)
		[1] 0.0004173274
		> calculate(0.040)*1e6
		[1] 417.3274
	\end{verbatim}
	Resposta: P(CF+ $|$ NEG) para Caucasianos  Não  Hispânicos vale 0.00042. Ou seja,
	417 indivíduos por milhão de habitantes.
	\subsection{b}
	A  probabilidade  P(CF+  $|$  NEG)  para  hispano-americanos. Esse  resultado
	representa aproximadamente quantos indivíduos por milhão de habitantes?
	\begin{verbatim}
		> calculate(0.017)
		[1] 0.0001732565
		> calculate(0.017)*1e6
		[1] 173.2565
	\end{verbatim}
	Resposta: P(CF+ $|$ NEG) para hispano-americanos vale 0.00017. Ou seja,
	173 indivíduos por milhão de habitantes.
	\subsection{c}
	A  probabilidade  P(CF+  $|$  NEG)  para  Afro-americanos. Esse  resultado
	representa aproximadamente quantos indivíduos por milhão de habitantes?
	\begin{verbatim}
		> calculate(0.016)
		[1] 0.0001629009
		> calculate(0.016)*1e6
		[1] 162.9009
	\end{verbatim}
	Resposta: P(CF+ $|$ NEG) para Afro-americanos vale 0.00016. Ou seja,
	163 indivíduos por milhão de habitantes.
	\subsection{d}
	A  probabilidade  P(CF+  $|$  NEG)  para  Judeus Asquenazes. Esse  resultado
	representa aproximadamente quantos indivíduos por milhão de habitantes?
	\begin{verbatim}
		> calculate(0.042)
		[1] 0.0004390991
		> calculate(0.042)*1e6
		[1] 439.0991
	\end{verbatim}
	Resposta: P(CF+ $|$ NEG) para Judeus Asquenazes vale 0.00043. Ou seja,
	439 indivíduos por milhão de habitantes.
	\subsection{e}
	A  probabilidade  P(CF+  $|$  NEG)  para  Asiático-americanos. Esse  resultado
	representa aproximadamente quantos indivíduos por milhão de habitantes?
	\begin{verbatim}
		> calculate(0.011)
		[1] 0.0001114339
		> calculate(0.011)*1e6
		[1] 111.4339
	\end{verbatim}
	Resposta: P(CF+ $|$ NEG) para Asiático-americanos vale 0.00011. Ou seja,
	111 indivíduos por milhão de habitantes.
	\section{Estatísticas Descritivas}
	Utilize os dados históricos da empresa Azul Linhas Aéreas Brasileiras S.A que foram obtidos
	a partir do site da ANAC [Agência Nacional de Aviação Civil] para responder aos itens
	solicitados.\\
	Os dados para resolver essa questão estão disponíveis em dois arquivos:\\
	i. ANAC\_\_\_AZUL\_\_\_4\_A\_7\_\_\_2019;\\
	ii. ANAC\_\_\_AZUL\_\_\_4\_A\_7\_\_\_2020\\
	Os dados históricos correspondem ao número de
	passageiros pagos do grupo de voo regular para os meses de Abril até Julho dos anos 2019 e
	2020, respectivamente.
	Determine:
	\subsection{a}
	Os números totais de Passageiros Pagos para cada um dos meses 4, 5, 6 e 7 de 2019
	e 2020. Apresente os dados na forma de uma tabela.
	\begin{verbatim}
		> df.2019 <- read.csv('ANAC___AZUL___4_A_7___2019.csv', sep=';')
		> df.2020 <- read.csv('ANAC___AZUL___4_A_7___2020.csv', sep=';')
		> summary(df.2019)
		 EMPRESA..SIGLA.    EMPRESA..NOME.     EMPRESA..NACIONALIDADE.      ANO      
		 Length:2126        Length:2126        Length:2126             Min.   :2019  
		 Class :character   Class :character   Class :character        1st Qu.:2019  
		 Mode  :character   Mode  :character   Mode  :character        Median :2019  
		                                                               Mean   :2019  
		                                                               3rd Qu.:2019  
		                                                               Max.   :2019  
		      MES        GRUPO.DE.VOO       PASSAGEIROS.PAGOS
		 Min.   :4.000   Length:2126        Min.   :    0    
		 1st Qu.:5.000   Class :character   1st Qu.:  464    
		 Median :6.000   Mode  :character   Median : 2082    
		 Mean   :5.497                      Mean   : 3977    
		 3rd Qu.:6.000                      3rd Qu.: 5294    
		 Max.   :7.000                      Max.   :30132    
		> summary(df.2020)
		 EMPRESA..SIGLA.    EMPRESA..NOME.     EMPRESA..NACIONALIDADE.      ANO      
		 Length:546         Length:546         Length:546              Min.   :2020  
		 Class :character   Class :character   Class :character        1st Qu.:2020  
		 Mode  :character   Mode  :character   Mode  :character        Median :2020  
		                                                               Mean   :2020  
		                                                               3rd Qu.:2020  
		                                                               Max.   :2020  
		                                                               NA's   :1     
		      MES        GRUPO.DE.VOO       PASSAGEIROS.PAGOS
		 Min.   :4.000   Length:546         Min.   :      0  
		 1st Qu.:5.000   Class :character   1st Qu.:    536  
		 Median :6.000   Mode  :character   Median :   1360  
		 Mean   :5.789                      Mean   :   5008  
		 3rd Qu.:7.000                      3rd Qu.:   3186  
		 Max.   :7.000                      Max.   :1367221  
		 NA's   :1

		> tot.2019 <- rep(0, 4)
		> tot.2020 <- rep(0, 4)
		> for (i in 4:7) {
		+     tot.2019[i - 3] <- sum(df.2019$PASSAGEIROS.PAGOS[which(df.2019$MES == i)])
		+     tot.2020[i - 3] <- sum(df.2020$PASSAGEIROS.PAGOS[which(df.2020$MES == i)])
		+ }
		> rbind(tot.2019, tot.2020)
		            [,1]    [,2]    [,3]    [,4]
		tot.2019 2059597 2128650 2018427 2247927
		tot.2020  163752  260970  368806  573693
	\end{verbatim}
	Dessa forma, tem-se a tabela \ref{tab:2a}.
	\begin{table}[H]
		\centering
		\begin{tabular}{ccc}
			\hline
			Mês & Total em 2019 & Total em 2020\\
			\hline
			Abril & 2059597 & 163752\\
			Maio & 2128650 & 260970\\
			Junho & 2018427 & 368806\\
			Julho & 2247927 & 573693\\
			\hline
		\end{tabular}
		\caption{Resposta para a 2a}
		\label{tab:2a}
	\end{table}
	\subsection{b}
	A média de Passageiros Pagos por mês, considerando todo o período analisado. Para
	isso, some todos os passageiros dos meses 4, 5, 6 e 7 e divida por quatro para os anos
	de 2019 e 2020, respectivamente. Compare.
	\begin{verbatim}
		> sum(df.2019$PASSAGEIROS.PAGOS)/4
		[1] 2113650
		> sum(df.2020$PASSAGEIROS.PAGOS[which(!is.na(df.2020$MES))])/4
		[1] 341805.2
	\end{verbatim}
	Resposta: Para 2019, 2113650. Para 2020, 341805,2. Houve uma queda significativa de 2019 para
	2020 de passageiros pagos, possivelmente em virtude da pandemia.
	\subsection{c}
	A moda de Passageiros Pagos, considerando todo o período entre os meses 4, 5, 6 e 7
	para os anos de 2019 e 2020, respectivamente. Desconsidere os casos em que
	Passageiros Pagos é igual a zero.
	Resposta: Para 2019, 465. Para 2020: 1041.
	\begin{verbatim}
		> modal <- function(vet) {
		+     vet_no_zero <- vet[which(vet != 0)]
		+     uniq <- unique(vet_no_zero)
		+     return (uniq[which.max(tabulate(match(vet_no_zero, uniq)))])
		+ }
		> modal(df.2019$PASSAGEIROS.PAGOS)
		[1] 465
		> modal(df.2020$PASSAGEIROS.PAGOS)
		[1] 1041
	\end{verbatim}
	\subsection{d}
	A mediana de Passageiros Pagos, considerando todo o período entre os meses 4, 5, 6 e
	7 para os anos de 2019 e 2020, respectivamente. Compare.
	\begin{verbatim}
		> median(df.2019$PASSAGEIROS.PAGOS)
		[1] 2081.5
		> median(df.2020$PASSAGEIROS.PAGOS)
		[1] 1359.5
	\end{verbatim}
	Resposta: Para 2019, 2081,5. Para 2020, 1359,5. A mediana, sendo metade abaixo menor
	e metade abaixo maior, representa de forma semelhante os efeitos da pandemia, como no quesito
	b.
	\subsection{e}
	O histograma de Passageiros Pagos para o Mês 4 [Abril] para os anos de 2019 e 2020,
	respectivamente. Compare. [O ideal aqui é fazer um histograma do tipo back-to-back].
	\begin{verbatim}
		> library(Hmisc)
		> vet.2019.abril <- df.2019$PASSAGEIROS.PAGOS[which(df.2019$MES == 4)]
		> vet.2020.abril <- df.2020$PASSAGEIROS.PAGOS[which(df.2020$MES == 4)]
		> jpeg('hist_plot.jpg')
		> histbackback(vet.2019.abril, vet.2020.abril, probability=TRUE, main='2.e')
		> dev.off()
	\end{verbatim}
	Resposta: Obteve-se a figura \ref{fig:backtoback}.
	\begin{figure}[H]
		\centering
		\includegraphics[width=7cm]{hist_plot.jpg}
		\caption{Histograma \textit{back-to-back} dos dados referentes a abril}
		\label{fig:backtoback}
	\end{figure}
	Ela indica como os valores de total de passageiros para 2020 diminuem quando comparados
	com 2019, coerente com o que se tem analisado dos outros dados.
	\subsection{f}
	O boxplot de Passageiros Pagos para o Mês 4 [Abril] para os anos de 2019 e 2020,
	respectivamente. Compare. [Apresente os boxplots em uma mesma figura].
	\begin{verbatim}
		> jpeg('box_plots.jpg')
		> boxplot(vet.2019.abril, vet.2020.abril, col='skyblue3', names=c('2019', '2020'))
		> dev.off()
	\end{verbatim}
	Resposta: Obteve-se a figura \ref{fig:boxplots}.
	\begin{figure}[H]
		\centering
		\includegraphics[width=7cm]{box_plots.jpg}
		\caption{\textit{Box plots} dos dados referentes a abril}
		\label{fig:boxplots}
	\end{figure}
	Verificam-se muitos \textit{outliers} em ambos os conjuntos de dados. Ademais, aqui se verifica
	ainda mais fortemente como os dados de 2019 superam os de 2020, com todos os quartis de 2019
	elevados quando comparados com 2020.
	\subsection{g}
	O gráfico de barras do total de Passageiros Pagos para os meses 4, 5, 6 e 7 para os anos
	de 2019 e 2020, respectivamente. Compare. [Faça os gráficos em uma mesma figura].
	\begin{verbatim}
		> jpeg('bar_plots.jpg')
		> barplot(rbind(tot.2019, tot.2020),
		+ xlab='Mês',
		+ col=c('red', 'green'),
		+ beside=TRUE
		+ )
		> legend('topleft',
		+ c('2019', '2020'),
		+ fill=c('red', 'green')
		+ )
		> dev.off()
	\end{verbatim}
	Resposta: Obteve-se a figura \ref{fig:barplots}.
	\begin{figure}[H]
		\centering
		\includegraphics[width=7cm]{bar_plots.jpg}
		\caption{\textit{Bar plots} para os dados mensais}
		\label{fig:barplots}
	\end{figure}
	Aqui se observa os aspectos discrepantes de forma mais evidente, comparando-se lado a lado
	os dados mensais de 2019 e 2020, respectivamente, pondo em cena fortemente a ideia de como a
	pandemia afetou as viagens para a companhia.
	\section{Combinatória}
	\subsection{a}
	Mostre passo a passo, usando álgebra, que a identidade abaixo é verdadeira.
	\begin{equation}
		\left(\begin{array}{c}
			n + 1\\k + 1
		\end{array}\right)
		=
		\left(\begin{array}{c}
			n\\k
		\end{array}\right)
		+
		\left(\begin{array}{c}
			n\\k + 1
		\end{array}\right)
	\end{equation}
	Demonstração:
	\begin{equation}
		\left(\begin{array}{c}
			n + 1 \\ k + 1
		\end{array}\right)
		= \frac{(n + 1)!}{(k + 1)!(n - k)!}
		= \frac{(n + 1)n!}{(n - k)(k + 1)!(n - k - 1)!)} \Rightarrow
	\end{equation}
	\begin{equation}
		\Rightarrow
		\left(\begin{array}{c}
			n + 1 \\ k + 1
		\end{array}\right)
		= \frac{((n - k) + (k + 1))n!}{(n - k)(k + 1)!(n - k - 1)!} \Rightarrow
	\end{equation}
	\begin{equation}
		\Rightarrow
		\left(\begin{array}{c}
			n + 1 \\ k + 1
		\end{array}\right)
		= \frac{n!}{(k + 1)!(n - k - 1)!} + \frac{n!}{k!(n - k)!} \Rightarrow
	\end{equation}
	\begin{equation}
		\Rightarrow
		\left(\begin{array}{c}
			n + 1 \\ k + 1
		\end{array}\right)
		=
		\left(\begin{array}{c}
			n\\k + 1
		\end{array}\right)
		+
		\left(\begin{array}{c}
			n\\k
		\end{array}\right)
	\end{equation}
	\subsection{b}
	Mostre passo a passo, usando álgebra, que a identidade abaixo é verdadeira.
	\begin{equation}
		\left(\begin{array}{c}
			2n \\ n
		\end{array}\right)
		+
		\left(\begin{array}{c}
			2n \\ n - 1
		\end{array}\right)
		=
		\frac{1}{2}\left(\begin{array}{c}
			2n + 2 \\ n + 1
		\end{array}\right)
	\end{equation}
	Demonstração:
	\begin{equation}
		\left(\begin{array}{c}
			2n \\ n
		\end{array}\right)
		+
		\left(\begin{array}{c}
			2n \\ n - 1
		\end{array}\right)
		=
		\frac{(2n)!}{n!n!} + \frac{(2n)!}{(n - 1)!(n + 1)!} \Rightarrow
	\end{equation}
	\begin{equation}
		\Rightarrow
		\left(\begin{array}{c}
			2n \\ n
		\end{array}\right)
		+
		\left(\begin{array}{c}
			2n \\ n - 1
		\end{array}\right)
		= \frac{(n + 1)(n + 1)(2n)!}{(n + 1)!(n + 1)!}
		+ \frac{(n + 1)n(2n)!}{(n + 1)!(n + 1)!} \Rightarrow
	\end{equation}
	\begin{equation}
		\Rightarrow
		\left(\begin{array}{c}
			2n \\ n
		\end{array}\right)
		+
		\left(\begin{array}{c}
			2n \\ n - 1
		\end{array}\right)
		= \frac{(2n + 1)(n + 1)(2n)!}{(n + 1)!(n + 1)!} \Rightarrow
	\end{equation}
	\begin{equation}
		\Rightarrow
		\left(\begin{array}{c}
			2n \\ n
		\end{array}\right)
		+
		\left(\begin{array}{c}
			2n \\ n - 1
		\end{array}\right)
		= \frac{1}{2}\frac{(2n + 2)(2n + 1)(2n)!}{(n + 1)!(n + 1)!}
		=
		\frac{1}{2}\left(\begin{array}{c}
			2n + 2 \\ n + 1
		\end{array}\right)
	\end{equation}
	\section{Combinatória e Probabilidade}
	\subsection{1}
	Nove linhas paralelas em um plano cruzam um conjunto de n linhas paralelas que vão
	em outra direção. As linhas formam um total de 360 paralelogramos, muitos dos quais se
	sobrepõem.
	\subsubsection{a}
	Encontre n [algebricamente].\\
	Solução:
	\begin{equation}
		\left(\begin{array}{c}
			9\\2
		\end{array}\right)
		\left(\begin{array}{c}
		    n\\2
		\end{array}\right)
		= 360 \Rightarrow
		\frac{n!}{(n - 2)!2!} = 10 \Rightarrow
	\end{equation}
	\begin{equation}
		\Rightarrow
		n(n - 1) = 20 \Rightarrow
		n^{2} - n - 20 = 0\Rightarrow
		n = \frac{1 \pm \sqrt{1 + 80}}{2}\Rightarrow
	\end{equation}
	\begin{equation}
		\Rightarrow
		n \in \{5, -4\} \Rightarrow
		n = 5
	\end{equation}
	\subsection{2}
	Um departamento de uma empresa consiste de 25 engenheiros, 15 analistas e 35
	assistentes. Um grupo de 6 desses profissionais é selecionado aleatoriamente desse
	departamento.
	\subsubsection{a}
	Encontre a probabilidade de que todos os membros do grupo sejam assistentes.
	\begin{verbatim}
		> espaco.amostral <- choose(25 + 15 + 35, 6)
		> evento <- choose(35, 6)
		> paste('p = ', evento/espaco.amostral)
		[1] "p =  0.00806100331471738"
	\end{verbatim}
	\subsubsection{b}
	\begin{verbatim}
		> evento <- choose(25, 2)*choose(15, 3)*choose(35, 1)
		> paste('p = ', evento/espaco.amostral)
		[1] "p =  0.0237262151211601"
	\end{verbatim}
\end{document}
