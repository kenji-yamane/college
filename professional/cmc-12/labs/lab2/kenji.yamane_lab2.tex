\documentclass{article}[twocolumn]
\usepackage[pdftex]{graphicx}
\usepackage[utf8]{inputenc}
\usepackage[brazil]{babel}
\usepackage{subfigure}
\usepackage{mathtools}
\usepackage{amsmath}
\usepackage{amssymb}
\usepackage{float}
\usepackage{tikz}

\title{Lab 2}
\author{Kenji Yamane}

\begin{document}
	\maketitle
	\section{Quest\~ao 1}
	O resultado advindo da simula\c{c}\~ao do controle frontal est\'a na Fig.
	\ref{fig:frontal}.
	\begin{figure}[H]
		\centering
		\includegraphics[width=7cm]{1.png}
		\caption{\textit{Output} do \textit{Simulink} relativo ao controlador frontal.}
		\label{fig:frontal}
	\end{figure}
	Como se percebe, a posi\c{c}\~ao est\'a de fato convergindo ao desejado, atingindo
	por volta de 63\% deste valor no tempo 0,15, em uma an\'alise a olho nu, o que tamb\'em
	condiz com o valor requisitado. Ademais, a fun\c{c}\~ao \'e claramente uma
	exponencial negativa, condizente com o fato do \'unico polo da fun\c{c}\~ao de
	transfer\^encia ser -Kx.
	\section{Quest\~ao 2}
	O conjunto de figuras \ref{fig:lateral1} compara os \textit{outputs} da simula\c{c}\~ao,
	y e $\psi$, para um valor de entrada 0,1 com rela\c{c}\~ao ao y.
	\begin{figure}[H]
		\centering
		\subfigure[Sa\'ida em y.]{
			\includegraphics[width=5cm]{lat_y1.png}
		}
		\subfigure[Sa\'ida em $\psi$.]{
			\includegraphics[width=5cm]{lat_psi1.png}
		}
		\caption{Para $y_r = 0,1$.}
		\label{fig:lateral1}
	\end{figure}
	A similaridade entre os dois gr\'aficos refor\c{c}a como a aproxima\c{c}\~ao \'e
	muito boa, mesmo para \^angulos grandes, como foi estipulado no roteiro.

	O conjunto de figuras \ref{fig:lateral3} compara os \textit{outputs} da simula\c{c}\~ao,
	y e $\psi$, para um valor de entrada 0,3 com rela\c{c}\~ao ao y.
	\begin{figure}[H]
		\centering
		\subfigure[Sa\'ida em y.]{
			\includegraphics[width=5cm]{lat_y3.png}
		}
		\subfigure[Sa\'ida em $\psi$.]{
			\includegraphics[width=5cm]{lat_psi3.png}
		}
		\caption{Para $y_r = 0,3$.}
		\label{fig:lateral3}
	\end{figure}
	Aqui diferen\c{c}as come\c{c}aram a aparecer, n\~ao em virtude da aproxima\c{c}\~ao,
	mas por causa da n\~ao linearidade, a satura\c{c}\~ao. O valor de pico neste caso
	aumentou para al\'em de 80 graus, de modo a revelar os efeitos da n\~ao linearidade.

	O conjunto de figuras \ref{fig:lateral5} compara os \textit{outputs} da simula\c{c}\~ao,
	y e $\psi$, para um valor de entrada 0,5 com rela\c{c}\~ao ao y. Aqui os gr\'aficos
	j\'a est\~ao bem diferentes.
	\begin{figure}[H]
		\centering
		\subfigure[Sa\'ida em y.]{
			\includegraphics[width=5cm]{lat_y5.png}
		}
		\subfigure[Sa\'ida em $\psi$.]{
			\includegraphics[width=5cm]{lat_psi5.png}
		}
		\caption{Para $y_r = 0,5$.}
		\label{fig:lateral5}
	\end{figure}
	Aqui se observa uma acentua\c{c}\~ao do que foi discutido na figura anterior, e vale
	notar que um atraso do modelo n\~ao linear para o linear j\'a existente na figura
	anterior tamb\'em se acentuou. Isso revela o porqu\^e de modelos n\~ao lineares n\~ao
	funcionarem com o controle cl\'assico. A habilidade do controle de se controlar os
	requisitos no tempo se perde. Por outro lado, no modelo linear, a fun\c{c}\~ao
	\texttt{pole} do \textit{MATLAB} revela que os polos dele possuem parte real negativa
	e parte imagin\'aria n\~ao nula, condizente com a oscila\c{c}\~ao amortecida
	gerada, mostrando como seu poder de resposta aos requisitos do tempo se mant\'em.

	Todavia, o modelo linear, tamb\'em n\~ao ter\'a efici\^encia quando integrado ao
	controlador frontal, pois a diferen\c{c}a de sinal decorrente de n\~ao haver cosseno
	aparece. Ser\'a necess\'ario aplicar a satura\c{c}\~ao no controlador completo, portanto.
	\section{Quest\~ao 3}
	Implementando-se o controlador completo, obteve-se os resultados expostos
	na Fig. \ref{fig:complete_control}.
	\begin{figure}[H]
		\centering
		\subfigure[De frente]{\includegraphics[width=5cm]{followline.png}}
		\subfigure[De tr\'as]{\includegraphics[width=5cm]{followline_neg.png}}
		\caption{Resultados obtidos com o controlador completo.}
		\label{fig:complete_control}
	\end{figure}
	Verifica-se que, apesar das diversas heur\'isticas (na verdade em virtude delas -.-),
	o controlador obteve desempenho excepcional, chegando nas posi\c{c}\~oes x e y
	desejadas com velocidade nula, e \textit{overshoot} pequeno.
\end{document}
