\documentclass{article}[twocolumn]
\usepackage[pdftex]{graphicx}
\usepackage[utf8]{inputenc}
\usepackage[brazil]{babel}
\usepackage{subfigure}
\usepackage{mathtools}
\usepackage{amsmath}
\usepackage{amssymb}
\usepackage{float}
\usepackage{tikz}

\title{Lista 2}
\author{Kenji Yamane}

\begin{document}
	\maketitle
	\newpage
	\section{Mapa de Hénon: mapa de convergência}
	Utilizou-se o seguinte procedimento em \textit{Python} para se gerar as imagens correspondentes
	ao quadrado de convergência.
	\begin{verbatim}
	def convergence_plot(func):
    	x_space = np.linspace(-2.5, 2.5, 700)
    	y_space = np.linspace(-2.5, 2.5, 700)
    	horizontal = []
    	vertical = []
    
        	for x in x_space:
            	for y in y_space:
                	input1 = x
                	input2 = y
        
            	for i in range(20):
                	input1, input2 = func(input1, input2)
    
                	if input1**2 + input2**2 > 100:
                    	horizontal.append(x)
                    	vertical.append(y)
                    	break
    	
    	fig, ax = plt.subplots()
    	plt.xticks([-2.5, 2.5])
    	plt.yticks([-2.5, 2.5])
    
    	ax.plot(horizontal, vertical, ',')
    	plt.show()
	\end{verbatim}
	Obteve-se assim, o conjunto de imagens \ref{fig:convergence}.
	\begin{figure}[H]
		\centering
		\begin{subfigure} {
			\includegraphics[width=5.5cm]{heron_convergence1.jpg}
		} \end{subfigure}
		\begin{subfigure} {
			\includegraphics[width=5.5cm]{heron_convergence2.jpg}
		} \end{subfigure}
		\caption{Figuras obtidas a partir do procedimento}
		\label{fig:convergence}
	\end{figure}
	\section{Bifurcação para o teorema de Heron}
	Aproveitou-se o procedimento criado na lista anterior para se gerar a figura requisitada.
	O resultado aparece na figura \ref{fig:bifurcation}.
	\begin{figure}[H]
		\centering
		\includegraphics[width=5.5cm]{bifurcation_diagram.jpg}
		\caption{Diagrama de bifurcação para o mapa de Heron}
		\label{fig:bifurcation}
	\end{figure}
	\section{Órbitas períodicas para o mapa de Heron}
	Criou-se a seguinte função para se gerar todas as figuras requisitadas, representativas
	de diversas órbitas periódicas para o mapa de Heron.
	\begin{verbatim}
	def plot_period(func, point_symbol, density):
    	convergence = False
    	global plot_iteration
    
    	while not convergence:
        	convergence = True
        	x = np.random.uniform(0, 1)
        	y = np.random.uniform(0, 1)
        	horizontal = []
        	vertical = []
        		
        	for i in range(100):
            	x, y = func(x, y)
            	if x**2 + y**2 > 100:
                	convergence = False
                	break
        
        	if convergence:
            	for i in range(density):
                	horizontal += [x]
                	vertical += [y]
                	x, y = func(x, y)
    	
    	fig, ax = plt.subplots()
    	ax.plot(horizontal, vertical, point_symbol)
    	plt.savefig('period' + str(plot_iteration) + '.jpg')
    	plot_iteration += 1
	\end{verbatim}
	Dessa forma, obteve-se o conjunto de imagens \ref{fig:periods}.
	\begin{figure}[H]
		\centering
		\begin{subfigure} {
			\includegraphics[width=5.4cm]{period1.jpg}
		} \end{subfigure}
		\begin{subfigure} {
			\includegraphics[width=5.4cm]{period2.jpg}
		} \end{subfigure}
		\begin{subfigure} {
			\includegraphics[width=5.4cm]{period3.jpg}
		} \end{subfigure}
		\begin{subfigure} {
			\includegraphics[width=5.4cm]{period4.jpg}
		} \end{subfigure}
		\begin{subfigure} {
			\includegraphics[width=5.4cm]{period5.jpg}
		} \end{subfigure}
		\begin{subfigure} {
			\includegraphics[width=5.4cm]{period6.jpg}
		} \end{subfigure}
		\caption{As órbitas periódicas para o mapa de heron}
		\label{fig:periods}
	\end{figure}
	\section{Órbita no pêndulo amortecido forçado}
	Implementando-se o algoritmo conhecido de runge-kutta para se resolver a EDO, pode-se gerar a figura requisitada:
	\begin{verbatim}
	angle = np.random.uniform(-pi, pi)
	velocity = np.random.uniform(-3, 4)
	time = 0
	horizontal = []
	vertical = []
	
	for i in range(500000):
    	horizontal += [angle%(2*pi) - pi]
    	vertical += [velocity]
    	for j in range(60):
        	angle, velocity = runge_kutta(angle, velocity, time, pi/30)
        	time += pi/30
	
	plt.plot(horizontal, vertical, ',')
	plt.savefig('pendulum_orbit.jpg')
	\end{verbatim}
	Com o procedimento, obteve-se a figura \ref{fig:chaos_pendulum}.
	\begin{figure}[H]
		\centering
		\includegraphics[width=5.5cm]{pendulum_orbit.jpg}
		\caption{Órbita obtida}
		\label{fig:chaos_pendulum}
	\end{figure}
\end{document}
