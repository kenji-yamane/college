\documentclass{article}[twocolumn]
\usepackage[pdftex]{graphicx}
\usepackage[utf8]{inputenc}
\usepackage[brazil]{babel}
\usepackage{subfigure}
\usepackage{mathtools}
\usepackage{amsmath}
\usepackage{amssymb}
\usepackage{float}
\usepackage{tikz}

\title{Lista 1}
\author{Kenji Yamane}

\begin{document}
	\maketitle
	\newpage
	\section{\textit{Cobweb Plots}}
	Utilizou-se a linguagem \textit{Python} para se gerar as figuras requeridas. O seguinte
	procedimento foi escrito para se gerar os dados:
	\begin{verbatim}
	def cobweb(starting_point, num_iterations, func):
    	"""
        	Creates necessary data to make a cobweb plot in matplotlib
        	:param starting_point: point in x axis to start the cobweb plot
        	:type starting_point: float
        	:param num_iterations: number of iterations, i.e. segments in cobweb plot
        	:type num_iterations: float
        	:param func: function whose cobweb plot data will be constructed
        	:type func: function
    	"""
    
    	horizontal = [starting_point]
    	vertical = [0]
    	start_point = [starting_point, 0]
    
    	for i in range(0, num_iterations):
        	if i%2 == 0:
            	end_point = [start_point[0], func(start_point[0])]
        	else:
            	end_point = [start_point[1], start_point[1]]
    
    	horizontal += [end_point[0]]
    	vertical += [end_point[1]]
    	start_point = end_point
    	
    	return horizontal, vertical
	\end{verbatim}
	Assim, com os dados, geraram-se as figuras com a biblioteca \textit{Matplotlib},
	obtendo-se dessa forma as figuras no conjunto \ref{fig:cobweb}.
	\begin{figure}[H]
		\centering
		\begin{subfigure} {
			\includegraphics[width=5.5cm]{cobweb1.jpg}
		} \end{subfigure}
		\begin{subfigure} {
			\includegraphics[width=5.5cm]{cobweb2.jpg}
		} \end{subfigure}
		\caption{\textit{Cobweb plots} requisitados}
		\label{fig:cobweb}
	\end{figure}
	\section{Estabilidade de pontos fixos}
	Deve-se provar o seguinte teorema:\\
	Seja f um mapa suave em $\mathbb{R}$, e p um ponto fixo do mesmo. Tem-se:
	\begin{equation}
		\left\{\begin{array}{c}
			|f'(p)| < 1 \quad \Rightarrow \quad $p é estável$\\
			|f'(p)| > 1 \quad \Rightarrow \quad $p é instável$
		\end{array}\right.
	\end{equation}
	Prova:\\
	Suponha $p \quad | \quad |f'(p)| < 1$. Logo:
	\begin{equation}
		\lim_{x \rightarrow p} \frac{|f(x) - f(p)|}{|x - p|} = |f'(p)|
	\end{equation}
	Assim, existe, para algum $N_{\epsilon}(p)$:
	\begin{equation}
		\frac{|f(x) - f(p)|}{|x - p|} \leqslant a
	\end{equation}
	Onde $|f'(p)| < a < 1$.\\
	Suponha que seja válido para a mesma $N_{\epsilon}(p)$:
	\begin{equation}
		\label{eq:kth_order}
		\frac{|f^{k}(x) - f^{k}(p)|}{|x - p|} \leqslant a^{k}
	\end{equation}
	Para $f^{k + 1}(x)$:
	\begin{equation}
		\lim_{x \rightarrow p} \frac{|f^{k + 1}(x) - f^{k + 1}(p)|}{|f^{k}(x) - f^{k}(p)|} =
		|f'^{k + 1}(p)|
	\end{equation}
	Como p é ponto fixo, $f'^{k + 1}(p) = f'(f^{k}(p)) = f'(p)$. Assim, existe para a mesma
	$N_{\epsilon}(p)$:
	\begin{equation}
		\label{eq:kplus1th_order}
		\frac{|f^{k + 1}(x) - f^{k + 1}(p)|}{|f^{k}(x) - f^{k}(p)|} \leqslant a
	\end{equation}
	Multiplicando \ref{eq:kth_order} e \ref{eq:kplus1th_order}\footnote{Válido pois envolve
	somente valores positivos}:
	\begin{equation}
		\frac{|f^{k + 1}(x) - f^{k + 1}(p)|}{|x - p|} \leqslant a^{k + 1}
	\end{equation}
	Por p ser ponto fixo, $f^{k}(p) = f(p)$ e, pela indução finita, tem-se:
	\begin{equation}
		|f^{k}(x) - p| \leqslant a^{k}|x - p|
	\end{equation}
	No limite de k para o infinito:
	\begin{equation}
		\lim_{k \rightarrow \infty}|f^{k}(x) - p| \leqslant \lim_{k \rightarrow \infty}a^{k}|x - p|
		\stackrel{a < 1}{\Rightarrow}
	\end{equation}
	\begin{equation}
		\stackrel{a < 1}{\Rightarrow}
		\lim_{k \rightarrow \infty}|f^{k}(x) - p| \leqslant 0
		\Rightarrow \lim_{k \rightarrow \infty}f^{k}(x) = p
	\end{equation}
	Dessa forma, todos os valores convergem para p, sendo p desta forma estável.\\
	Para o caso em que $|f'(x)| > 1$, partimos de:
	\begin{equation}
		\lim_{x \rightarrow p} \frac{|f(x) - f(p)|}{|x - p|} = |f'(p)|
	\end{equation}
	Sabe-se, da definição de limite, que pode-se limitar x de tal forma que se aproxima o lado
	esquerdo do lado direito o tanto quanto se queira. Dessa forma, como $|f'(x)| > 1$, existe
	vizinhança $N_{\epsilon}(p)$ tal que:
	\begin{equation}
		\frac{|f(x) - f(p)|}{|x - p|} > 1
	\end{equation}
	Analogamente ao caso em que $|f(x)| < 1$, tem-se:
	\begin{equation}
		\frac{|f^{k}(x) - f^{k}(p)|}{|x - p|} > 1 \Rightarrow
		|f^{k}(x) - p| > |x - p|
	\end{equation}
	Dessa forma, tem-se que, mesmo eventualmente, o valor mapeado se afasta de p. Como isso é válido
	para toda a vizinhança $N_{\epsilon}(p)$, eventualmente todos os pontos diferentes de p
	sairão de $N_{\epsilon}(p)$. Assim, p é instável.
	\section{Exercício T1.5}
	Deve-se determinar os pontos de período 2 da seguinte função:
	\begin{equation}
		f(x) = 2x^{2} - 5x
	\end{equation}
	\begin{equation}
		f(f(x)) = x \Rightarrow 2(2x^{2} - 5x)^{2} - 5(2x^{2} - 5x) = x \Rightarrow
	\end{equation}
	\begin{equation}
		\Rightarrow 2(4x^{4} - 20x^{3} + 25x^{2}) - 10x^{2} + 25x = x \Rightarrow
	\end{equation}
	\begin{equation}
		\Rightarrow 8x^{4} - 40x^{3} + 40x^{2} + 24x = 0 \Rightarrow
	\end{equation}
	\begin{equation}
		\Rightarrow x^{4} - 5x^{3} + 5x^{2} + 3x = 0 \Rightarrow
	\end{equation}
	\begin{equation}
		\Rightarrow x = 0 \quad ou \quad x^{3} - 5x^{2} + 5x + 3 = 0
	\end{equation}
	Por inspeção, 3 é raiz. Logo:
	\begin{equation}
		\Rightarrow x = 0 \quad ou \quad (x - 3)(x^{2} - 2x - 1) = 0 \Rightarrow
	\end{equation}
	\begin{equation}
		\Rightarrow x \in \{0, 3\} \quad ou \quad x^{2} - 2x - 1 = 0
	\end{equation}
	\begin{equation}
		\Rightarrow x \in \{0, 3\} \quad ou \quad x = \frac{2 \pm \sqrt{4 + 4}}{2} \Rightarrow
	\end{equation}
	\begin{equation}
		\Rightarrow x \in \{0, 3, 1 - \sqrt{2}, 1 + \sqrt{2}\}
	\end{equation}
	Como 0 e 3 são pontos fixos de f, não são pontos de período 2. A resposta é portanto,
	$\{1 - \sqrt{2}, 1 + \sqrt{2}\}$.
	\section{Diagramas de bifurcação}
	Escreveu-se o seguinte procedimento para gerar os diagramas requisitados.
	\begin{verbatim}
	def bifurcation_diagram(start, end, figname):
    	"""
        	Constructs the bifurcation diagram for map_function1
        	:param start: point in x axis to start plotting
        	:type start: float
        	:param end: point in x axis to end plotting
        	:type end: float
        	:param figname: name of the figure to be saved
        	:type end: float
    	"""
    	a_space = np.linspace(start, end, 500)
    	horizontal = []
    	vertical = []
	
    	for a in a_space:
        	point = np.random.uniform(0, 1)
        	for i in range(100):
            	point = map_function1(a, point)
		
        	for i in range(700):
            	horizontal.append(a)
            	vertical.append(point)
            	point = map_function1(a, point)
	
    	fig, ax = plt.subplots()
    	ax.plot(horizontal, vertical, ',')
    	plt.savefig(figname)
	\end{verbatim}
	Obteve-se dessa forma o conjunto mostrado em \ref{fig:bifurcation}.
	\begin{figure}[H]
		\centering
		\begin{subfigure}{
			\includegraphics[width=5.5cm]{bifurcation1.jpg}
		}\end{subfigure}
		\begin{subfigure}{
			\includegraphics[width=5.5cm]{bifurcation2.jpg}
		}\end{subfigure}
		\begin{subfigure}{
			\includegraphics[width=5.5cm]{bifurcation3.jpg}
		}\end{subfigure}
		\caption{Diagramas de bifurcação obtidos}
		\label{fig:bifurcation}
	\end{figure}
	\section{\textit{Cobweb plot caótico}}
	Com a mesma função mostrada no início, obteve-se a seguinte imagem.
	\begin{figure}[H]
		\centering
		\includegraphics[width=5.5cm]{cobweb3.jpg}
		\caption{\textit{Cobweb plot} obtido para $a = 3.86$}
	\end{figure}
\end{document}
