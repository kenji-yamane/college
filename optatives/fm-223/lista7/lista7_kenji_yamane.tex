\documentclass{article}[twocolumn]
\usepackage[pdftex]{graphicx}
\usepackage[utf8]{inputenc}
\usepackage[brazil]{babel}
\usepackage{subfigure}
\usepackage{mathtools}
\usepackage{amsmath}
\usepackage{amssymb}
\usepackage{float}
\usepackage{tikz}

\title{Lista 7}
\author{Kenji Yamane}

\begin{document}
	\maketitle
	\section{Mapa de \textit{Tinkerbell}}
	Utilizou-se o seguinte c\'odigo para se determinar a \'orbita requerida.
	\begin{verbatim}
import numpy as np
import matplotlib.pyplot as plt

ORBIT_LENGTH = 1000000
DISCARDED_POINTS = 100

def tinkerbell_map(x, y, c4):
    return x**2 - y**2 - 0.3*x - 0.6*y, 2*x*y + 2*x + c4*y

horizontal = []
vertical = []

x_travel = 0.1
y_travel = 0.1

for i in range(ORBIT_LENGTH):
    x_travel, y_travel = tinkerbell_map(x_travel, y_travel, 0.7)
    if i > DISCARDED_POINTS:
        horizontal.append(x_travel)
        vertical.append(y_travel)

plt.plot(horizontal, vertical, ',')
plt.savefig('./tinkerbell/bigger.jpg')

	\end{verbatim}
	Utilizou-se uma \'orbita de tamanho de um milh\~ao de pontos, descartando-se os 100
	primeiros pontos gerados. Para o valor de 0,5 para o c4 e valores pr\'oximos dele,
	de fato obtiveram-se formas aproximadamente ovais, como mostrado a seguir.
	\begin{figure}[H]
		\centering
		\subfigure[\textit{Tinkerbell} para $c_{4}$ = 0,4]{
			\includegraphics[width=5.5cm]{tinkerbell/slightly_smaller.jpg}
		}
		\subfigure[\textit{Tinkerbell} para $c_{4}$ = 0,5]{
			\includegraphics[width=5.5cm]{tinkerbell/normal.jpg}
		}
		\subfigure[\textit{Tinkerbell} para $c_{4}$ = 0,6]{
			\includegraphics[width=5.5cm]{tinkerbell/slightly_bigger.jpg}
		}
		\caption{\'Orbitas obtidas para valores pr\'oximos de 0,5 para o $c_{4}$.}
	\end{figure}
	Para valores maiores que 0,5, obteve-se um formato parecido com uma espiral e para valores
	menores que 0,5, obteve-se novamente ciclos quasiperi\'odico, com formatos mais diferentes
	de um oval, mostrados a seguir.
	\begin{figure}[H]
		\centering
		\subfigure[\textit{Tinkerbell} para $c_{4}$ = 0]{
			\includegraphics[width=5.5cm]{tinkerbell/smaller.jpg}
		}
		\subfigure[\textit{Tinkerbell} para $c_{4}$ = 0,7]{
			\includegraphics[width=5.5cm]{tinkerbell/bigger.jpg}
		}
		\caption{\'Orbitas obtidas para valores mais distantes de 0,5 para o $c_{4}$.}
	\end{figure}
	Para valores bem maiores ou bem menores que 0,5 para $c_{4}$, problemas de
	\textit{overflow} come\c{c}aram a aparecer, o que sugere que bacias de atra\c{c}\~ao para
	o infinito come\c{c}am a aparecer.
	\section{C\'alculo num\'erico da derivada}
	Utilizando a fun\c{c}\~ao arco tangente:
	\begin{equation}
		f(x) = arctan(x)
		\nonumber
	\end{equation}
	Cuja derivada anal\'itica corresponde a:
	\begin{equation}
		f'(x) = \frac{1}{1 + x^2}
		\nonumber
	\end{equation}
	Aproximou-se numericamente o valor desta derivada para x = 1 por diferen\c{c}as
	finitas progressivas, e calculou-se o erro:
	\begin{equation}
		E(\partial x) = \left|
		f'(1) - \frac{f(1 + \partial x) - f(x)}{\partial x}
		\right|
		\nonumber
	\end{equation}
	Para um valor de $\partial x = 10^{-i}, \quad i \in \{0, 1, 2, ..., 16\}$, podendo-se
	ent\~ao gerar um gr\'afico E por $\partial x$. Utilizou-se o seguinte c\'odigo:
	\begin{verbatim}
from math import atan
import matplotlib.pyplot as plt

horizontal = []
vertical = []

for i in range(17):
    dx = 10**(-i)
    horizontal.append(dx)
    vertical.append(abs(1/2 - (atan(1 + dx) - atan(1))/dx))

fig, ax = plt.subplots()

ax.plot(horizontal, vertical, '.')
plt.xlabel('dx')
plt.ylabel('E')
ax.set_xscale('log')
plt.savefig('derivative.jpg')

	\end{verbatim}
	Com o qual p\^ode-se gerar o seguinte gr\'afico.
	\begin{figure}[H]
		\centering
		\includegraphics[width=6cm]{derivative.jpg}
		\caption{Valor obtido pela aproxima\c{c}\~ao para diferentes $\partial x$.}
	\end{figure}
	Onde se observa claramente uma diverg\^encia com a realidade, derivada anal\'itica, para
	valores muito grandes e valores muito pequenos de $\partial x$.
	
	O erro associado \`a valores muito grandes \'e explicado simplesmente pela aproxima\c{c}\~ao
	necessitar de valores bem pequenos para ser v\'alida, um $\partial x$, por defini\c{c}\~ao
	\'e infinitamente pequeno.

	O erro associado \`a valores muito pequenos, entretanto, \'e explicado por outra
	\'optica, dado que um valor menor deveria ser mais preciso. Ele \'e um sintoma do erro
	intransigente das opera\c{c}\~oes de uma m\'aquina, a qual precisa aproximar um n\'umero real
	\`as vezes, gerando um erro associado. Quando se trabalha com valores muito pequenos,
	faz sentido esta inabilidade aparecer.
	\section{Eps}
	Utilizou-se o seguinte c\'odigo em \textit{Python} para se calcular o eps da minha m\'aquina,
	utilizando algoritmo de busca bin\'aria.
	\begin{verbatim}
import numpy as np

lesser = 1e-20
bigger = 1

NUM_ITERATIONS = 100

for i in range(NUM_ITERATIONS):
    middle = (lesser + bigger)/2
    
    if 1 + middle == 1:
        lesser = middle
    else:
        bigger = middle

print(bigger)
print(np.finfo(float).eps)
	\end{verbatim}
	Obteve-se o valor de $1,1102230246251588.10^{-16}$ numericamente com o algoritmo de busca
	bin\'aria e o valor de $2,220446049250313.10^{-16}$ com a fun\c{c}\~ao do pacote
	\textit{numpy}, claramente o dobro. Somando-se o valor obtido numericamente com 1,
	de fato, resulta num valor maior que 1, por\'em errado, como se estivesse somado
	com o valor obtido por \textit{numpy}. Pesquisou-se no site do \textit{Matlab} e ele
	tamb\'em indica o segundo valor como o valor de eps, dizendo ser correspondente
	a $2^{-52}$. O fato de valores at\'e metade ainda resultarem em um valor diferente de
	1 quando somados a 1 pode ser explicado portanto levando em considera\c{c}\~ao que
	apesar deles serem menores que o \'ultimo bit dispon\'ivel, s\~ao aproximados
	na conta da m\'aquina para este bit.
	\section{N\'umeros de \textit{Lyapunov}}
	Utilizou-se o seguinte c\'odigo feito com base no algoritmo fornecido para
	calcular os expoentes de \textit{Lyapunov} assim como a dimens\~ao de \textit{Lyapunov}.
	\begin{verbatim}
import numpy as np
from math import log

ORBIT_LENGTH = 10000
MAP_DIMENSION = 2

def henon_map(vet):
    return np.array([1.4 - vet[0]**2 + 0.3*vet[1], vet[0]])

def henon_jacobian(vet):
    return np.array([[-2*vet[0], 0.3], [1, 0]])

eye = np.eye(MAP_DIMENSION)
w = []
for i in range(MAP_DIMENSION):
    w.append(eye[i])
v = np.array([0, 0])
lyapunov_exponents = np.zeros(MAP_DIMENSION)

for i in range(ORBIT_LENGTH):
    z = []
    for j in range(MAP_DIMENSION):
        z.append(np.matmul(henon_jacobian(v), w[j]))

    y = []
    for j in range(MAP_DIMENSION):
        y.append(z[j])
        for k in range(j):
            y[j] -= y[k]*np.dot(z[j], y[k])/np.dot(y[k], y[k])
	
    for j in range(MAP_DIMENSION):
        lyapunov_exponents[j] += log(np.linalg.norm(y[j]))

    for j in range(MAP_DIMENSION):
        w[j] = y[j]/np.linalg.norm(y[j])

    v = henon_map(v)

for i in range(MAP_DIMENSION):
    lyapunov_exponents[i] /= ORBIT_LENGTH

for i in range(MAP_DIMENSION):
    print(lyapunov_exponents[i])

print('dimension:', 1 + lyapunov_exponents[0]/abs(lyapunov_exponents[1]))

	\end{verbatim}
	Obtendo-se com este c\'odigo valores de 0.413 e -1.617 para os expoentes de
	\textit{Lyapunov}, os quais est\~ao perfeitamente condizentes com os valores indicados
	no livro, estes os quais s\~ao basicamentes os encontrados pelo c\'odigo porem truncados
	na segunda casa decimal.

	A dimens\~ao de \textit{Lyapunov} obtida foi de 1,26, bem mais pr\'oxima do valor
	encontrado pelo livro, 1,27, do que o valor encontrado na lista passada, 1,38, o que
	sugere como o valor do livro est\'a mais pr\'oximo do valor real.
\end{document}
