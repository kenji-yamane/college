\documentclass{article}[twocolumn]
\usepackage[pdftex]{graphicx}
\usepackage[utf8]{inputenc}
\usepackage[brazil]{babel}
\usepackage{subfigure}
\usepackage{mathtools}
\usepackage{amsmath}
\usepackage{amssymb}
\usepackage{float}
\usepackage{tikz}

\title{Lista 8}
\author{Kenji Yamane}

\begin{document}
	\maketitle
	\section{Medida natural de um mapa linear por partes}
	Calculou-se a fun\c{c}\~ao p cuja integral resulta na medida
	natural para o seguinte mapa:
	\begin{equation}
		f(x) = \left\{\begin{array}{ll}
			2 + \sqrt{2}(x - 1) & se \quad 0 \leq x \leq \frac{2 - \sqrt{2}}{2}\\
			\sqrt{2}(1 - x) & se \quad \frac{2 - \sqrt{2}}{2} \leq x \leq 1
		\end{array}\right.
		\nonumber
	\end{equation}
	Utilizando o seguinte c\'odigo em \textit{Python}:
	\begin{verbatim}
import numpy as np
from math import sqrt
import matplotlib.pyplot as plt

NUM_BINS = 100
ORBIT_LEN = 300000000
CUTS = 1

def natural_measure(func, start, end):
    bin_length = (end - start)/NUM_BINS
    bins = np.zeros(NUM_BINS)
    orbit_point = (start + end)/2
    orbit_point = (start + orbit_point)/2
    orbit_point = (start + orbit_point)/2

    for i in range(ORBIT_LEN):
        orbit_point = func(orbit_point)
        		
        to_start = orbit_point - start
        to_bins = int(to_start/bin_length)
        bins[to_bins] = bins[to_bins] + 1
	
    bins /= ORBIT_LEN # natural measure
    bins /= bin_length # invariant measure
    
    return bins

def critical_map(x):
    c = 1 - sqrt(2)/2
    if x < c:
        return 2 + sqrt(2)*(x - 1)
    else:
        return sqrt(2)*(1 - x)

def logistic_map(x):
    return 4*x*(1 - x)

x = np.linspace(0, 1, NUM_BINS)
y = natural_measure(critical_map, 0, 1)
x = x[CUTS:-CUTS]
y = y[CUTS:-CUTS]
plt.ylim(0, 2.5)
plt.plot(x, y, '.')
plt.savefig('critical_measure.jpg')

	\end{verbatim}
	Obtendo-se desta forma a seguinte imagem, posta ao lado da imagem obtida no livro
	do \textit{Alligood}:
	\begin{figure}[H]
		\centering
		\subfigure[Fun\c{c}\~ao p obtida pelo c\'odigo.]{
			\includegraphics[width=5.5cm]{measures/critical_measure.jpg}
		}
		\subfigure[Correspondente apresentado no livro.]{
			\includegraphics[width=5.5cm]{measures/critical_book.jpg}
		}
		\caption{Densidade de medida.}
	\end{figure}
	Inicialmente o ru\'ido associado estava bem maior por\'em notou-se que quanto
	menos intervalos e maior a \'orbita, maior era a precis\~ao obtida, resultando nesta
	imagem, cuja semelhan\c{c}a com a do livro de fato \'e grande. \'E razo\'avel assumir que
	no limite da \'orbita ao infinito, o resultado do c\'ogido iria convergir ao obtido
	pelo livro.
	\section{Medida natural do mapa log\'istico}
	Utilizou-se o mesmo c\'odigo, alterando par\^ametros como o tamanho da \'orbita,
	para se calcular a densidade de medida do mapa log\'istico com a = 4:
	\begin{equation}
		G(x) = 4x(1 - x)
		\nonumber
	\end{equation}
	Obtendo-se assim o seguinte gr\'afico, posto ao lado do obtido pelo livro do
	\textit{Alligood}:
	\begin{figure}[H]
		\centering
		\subfigure[Fun\c{c}\~ao p obtida pelo c\'odigo.]{
			\includegraphics[width=5.5cm]{measures/logistic_measure.jpg}
		}
		\subfigure[Correspondente apresentado no livro.]{
			\includegraphics[width=5.5cm]{measures/logistic_book.jpg}
		}
		\caption{Densidade de medida.}
	\end{figure}
	Percebe-se como ele est\'a bem mais pr\'oximo da curva te\'orica em rela\c{c}\~ao
	ao mapa anterior, e vale enfatizar como foi necess\'ario uma \'orbita menor que a anterior
	para se obter esta imagem, podendo-se aumentar tamb\'em o n\'umero de intervalos usados.
	Isso provavelmente decorre do fato desta fun\c{c}\~ao ser suave, e portanto mais bem
	comportada em termos de derivada e \'orbitas.
	\section{Exerc\'icio 7.1}
	\subsection{a}
	\begin{equation}
		\ddot{x} + kx = 0
		\nonumber
	\end{equation}
	\begin{equation}
		y = \dot{x} \Rightarrow \dot{y} = \ddot{x} \Rightarrow
		\nonumber
	\end{equation}
	\begin{equation}
		\Rightarrow \left\{\begin{array}{lll}
			\dot{x} & = & y\\
			\dot{y} & = & -kx
		\end{array}\right.
		\nonumber
	\end{equation}
	Assim a matriz associada \'e:
	\begin{equation}
		A = \left[\begin{array}{cc}
			0 & 1\\
			-k & 0
		\end{array}\right] \Rightarrow
		det(A - I\lambda) = \left|\begin{array}{cc}
			-\lambda & 1\\
			-k & -\lambda
		\end{array}\right| \Rightarrow
		\nonumber
	\end{equation}
	\begin{equation}
		\Rightarrow det(A - I\lambda) = \lambda^2 + k
		\nonumber
	\end{equation}
	\subsection{b}
	Caso k seja positivo, os autovalores ser\~ao puramente imagin\'arios,
	o que implica em curvas peri\'odicas ao redor da origem. Utilizando
	o seguinte c\'odigo:
	\begin{verbatim}
import numpy as np
from math import cos, sin, pi
import matplotlib.pyplot as plt

TRAJECTORY_LENGTH = 7000
TIME_STEP = 0.001
ARROW_RESOLUTION = 0.0004
ARROW_STEP = 1000

def rotation_mtx(theta):
    row_1 = [cos(theta), -sin(theta)]
    row_2 = [sin(theta), cos(theta)]
    
    return np.array([row_1, row_2])

def ode_law(x):
    k = 1
    return np.array([x[1], -k*x[0]])

def runge_kutta_step(x):
    x1 = TIME_STEP*ode_law(x)
    x2 = TIME_STEP*ode_law(x + x1/2)
    
    return x + x2

horizontal = []
vertical = []

def draw_arrow(point, direction):
    original_point = np.array(point)
    direction /= np.linalg.norm(direction)
    
    end_1 = np.matmul(rotation_mtx(3*pi/4), direction)
    end_2 = np.matmul(rotation_mtx(-3*pi/4), direction)
    
    for i in range(100):
        point += end_1*ARROW_RESOLUTION
        horizontal.append(point[0])
        vertical.append(point[1])
    	
    point = original_point
    for i in range(100):
        point += end_2*ARROW_RESOLUTION
        horizontal.append(point[0])
        vertical.append(point[1])

for i in range(9):
    orbit_point = np.array([0, (i + 1)/10])
    direction = ode_law(orbit_point)
    arrows = 0

    for j in range(TRAJECTORY_LENGTH):
        if arrows == ARROW_STEP:
            direction = ode_law(orbit_point)
            draw_arrow(np.array(orbit_point), direction)
            arrows = 0
    orbit_point = runge_kutta_step(orbit_point)
    
    horizontal.append(orbit_point[0])
    vertical.append(orbit_point[1])
    
    arrows += 1

fig, ax = plt.subplots()
ax.spines['left'].set_position('center')
ax.spines['right'].set_color('none')
ax.spines['bottom'].set_position('center')
ax.spines['top'].set_color('none')

ax.plot(horizontal, vertical, 'k,')
plt.savefig('harmonic_mhs.jpg')

	\end{verbatim}
	Obt\'em-se a seguinte figura para o plano de fases da EDO, para k = 1.
	\begin{figure}[H]
		\centering
		\includegraphics[width=6cm]{phase_planes/harmonic_mhs.jpg}
		\caption{Plano de fases para a EDO harm\^onica com k positivo.}
	\end{figure}
	O que corrobora o resultado previsto teoricamente.
	
	Para k negativo:
	\begin{equation}
		\lambda = -\sigma^2 \Rightarrow det(A - I\lambda) = \lambda^2 + k \Rightarrow
		\lambda^2 - \sigma^2 = 0 \Rightarrow \lambda \in \{\sigma, -\sigma\}
		\nonumber
	\end{equation}
	Sendo portanto um ponto de sela. Encontrando o autovetor de $\sigma$:
	\begin{equation}
		\left[\begin{array}{cc}
			0 & 1\\
			\sigma^2 & 0
		\end{array}\right] \left[\begin{array}{c}
			a\\
			b
		\end{array}\right] = \left[\begin{array}{c}
			\sigma a\\
			\sigma b
		\end{array}\right] \Rightarrow
		b = \sigma a \Rightarrow v_{\sigma} = \left[\begin{array}{c}
			1\\
			\sigma
		\end{array}\right]
		\nonumber
	\end{equation}
	Analogamente, para $-\sigma$:
	\begin{equation}
		\Rightarrow v_{-\sigma} = \left[\begin{array}{c}
			1\\
			-\sigma
		\end{array}\right]
		\nonumber
	\end{equation}
	Adaptando o mesmo c\'odigo para funcionar com k = -1, obteve-se a seguinte figura:
	\begin{figure}[H]
		\centering
		\includegraphics[width=6cm]{phase_planes/exponential_msh.jpg}
		\caption{Plano de fases para a EDO harm\^onica com k negativo.}
	\end{figure}
	O que novamente corrobora com o resultado previsto numericamente. Em virtude da forma
	como o c\'odigo foi escrito, a densidade de flechas indica a rela\c{c}\~ao com o tempo.
	Enquanto na primeira figura esta \'e constante, nesta de agora os efeitos exponenciais
	est\~ao vis\'iveis.
	
	Para k = 0:
	\begin{equation}
		\dot{y} = kx \Rightarrow \dot{y} = 0 \Rightarrow y = c \Rightarrow x = ct + d
		\nonumber
	\end{equation}
	Portanto, para valores positivos de y, x aumenta, para valores negativos de y, x
	diminui, e toda a reta y = 0 corresponde a \textit{equilibria}.

	Inserindo como pontos iniciais como sendo da reta x = 0 no c\'odigo usado anteriormente:
	\begin{figure}[H]
		\centering
		\includegraphics[width=6cm]{phase_planes/zero_mhs.jpg}
		\caption{Plano de fases para a EDO harm\^onica com k nulo.}
	\end{figure}
	A densidade de flechas corretamente mostra que quanto maior o m\'odulo de y,
	maior o m\'odulo da taxa de varia\c{c}\~ao de x.
	\section{Exerc\'icio 7.3}
	Os pontos de equil\'ibrio correspondem \`aqueles tais que \textbf{f}(\textbf{x}) = 0.
	Assim:
	\begin{equation}
		\left\{\begin{array}{c}
			2x - y = 0\\
			x^2 + 4y = 0
		\end{array}\right. \Rightarrow
		y = 2x \Rightarrow x^2 + 8x = 0 \Rightarrow
		\nonumber
	\end{equation}
	\begin{equation}
		\Rightarrow x \in \{0, -8\}
		\nonumber
	\end{equation}
	Como $y = 2x$, os pontos de equil\'ibrio s\~ao (0, 0) e (-8, -16).
	\begin{equation}
		D_f = \left[\begin{array}{cc}
			\frac{\partial f_1}{\partial x} & \frac{\partial f_1}{\partial y}\\
			\frac{\partial f_2}{\partial x} & \frac{\partial f_2}{\partial y}
		\end{array}\right] \Rightarrow
		D_f = \left[\begin{array}{cc}
			2 & -1\\
			2x & 4
		\end{array}\right]
		\nonumber
	\end{equation}
	Para o ponto (0, 0), a matriz $D_f$ \'e triangular superior e portanto seus
	autovalores s\~ao os elementos da diagonal principal, 2 e 4. Como ambos s\~ao
	positivos, (0, 0) corresponde a um ponto inst\'avel.

	Para o ponto (-8, -16):
	\begin{equation}
		D_f(-8, -16) = \left[\begin{array}{cc}
			2 & -1\\
			-16 & 4
		\end{array}\right] \Rightarrow
		det(D_f(-8, -16) - \lambda I) = \left|\begin{array}{cc}
			2 - \lambda & -1\\
			-16 & 4 - \lambda
		\end{array}\right| \Rightarrow
		\nonumber
	\end{equation}
	\begin{equation}
		\Rightarrow det(D_f(-8, -16) - \lambda I) = \lambda^2 - 6\lambda - 8 = 0
		\Rightarrow \lambda = \frac{6 \pm \sqrt{36 + 48}}{2} \Rightarrow
		\nonumber
	\end{equation}
	\begin{equation}
		\lambda = 3 \pm \sqrt{21}
		\nonumber
	\end{equation}
	Como pelo menos um dos autovalores \'e positivo e nenhum \'e nulo, (-8, -16)
	tamb\'em \'e inst\'avel.
	\section{Exerc\'icio 7.9}
	\begin{equation}
		\ddot{x} + \dot{x} + x + x^3 = 0
		\nonumber
	\end{equation}
	O que \'e equivalente a:
	\begin{equation}
		\left\{\begin{array}{l}
			\dot{x} = y\\
			\dot{y} = -x - x^3 - y\\
		\end{array}\right.
		\nonumber
	\end{equation}
	\begin{equation}
		D_f = \left[\begin{array}{cc}
			\frac{\partial f_1}{\partial x} & \frac{\partial f_1}{\partial y}\\
			\frac{\partial f_2}{\partial x} & \frac{\partial f_2}{\partial y}
		\end{array}\right] \Rightarrow
		D_f = \left[\begin{array}{cc}
			0 & 1\\
			-1 - 3x^2 & -1
		\end{array}\right]
		\nonumber
	\end{equation}
	Para o ponto (0, 0):
	\begin{equation}
		D_f = \left[\begin{array}{cc}
			0 & 1\\
			-1 & -1
		\end{array}\right] \Rightarrow
		det(D_f - \lambda I) = \left|\begin{array}{cc}
			-\lambda & 1\\
			-1 & -1 - \lambda
		\end{array}\right| \Rightarrow
		\nonumber
	\end{equation}
	\begin{equation}
		\Rightarrow
		det(D_f - \lambda I) = \lambda^2 + \lambda + 1 = 0 \Rightarrow
		\lambda = \frac{-1 \pm \sqrt{1 - 4}}{2} \Rightarrow
		\nonumber
	\end{equation}
	\begin{equation}
		\lambda \in \{cis(\frac{2\pi}{3}), cis(\frac{4\pi}{3})\}
		\nonumber
	\end{equation}
	Assim, todos os autovalores da fun\c{c}\~ao possuem parte real estritamente negativa
	e portanto (0, 0) \'e um ponto assintoticamente est\'avel.
	\section{Exerc\'icio 7.16}
	A EDO \'e equivalente a:
	\begin{equation}
		\left\{\begin{array}{l}
			\dot{x} = y\\
			\dot{y} = x - ax^3
		\end{array}\right.
		\nonumber
	\end{equation}
	Pontos de equil\'ibrio devem resultar no vetor nulo:
	\begin{equation}
		\left\{\begin{array}{l}
			y = 0\\
			x - ax^3 = 0
		\end{array}\right. \Rightarrow x(1 - ax^2) = 0
		\nonumber
	\end{equation}
	Assim (0, 0) \'e ponto de equil\'ibrio e, se a for positivo, $(\pm\frac{1}{\sqrt{a}}, 0)$
	tamb\'em s\~ao.
	\begin{equation}
		D_f = \left[\begin{array}{cc}
			\frac{\partial f_1}{\partial x} & \frac{\partial f_1}{\partial y}\\
			\frac{\partial f_2}{\partial x} & \frac{\partial f_2}{\partial y}
		\end{array}\right] \Rightarrow
		D_f = \left[\begin{array}{cc}
			0 & 1\\
			1 - 3ax^2 & 0
		\end{array}\right] \Rightarrow
		\nonumber
	\end{equation}
	\begin{equation}
		\Rightarrow det(D_f - \lambda I) = \lambda^2 - 1 + 3ax^2 = 0
		\label{eq:characteristic}
	\end{equation}
	Assim, para (0, 0), os autovalores s\~ao 1 e -1. Como pelo menos um dos autovalores \'e
	positivo, (0, 0) \'e inst\'avel. Com rela\c{c}\~ao a seus autovetores:
	\begin{equation}
		\left[\begin{array}{cc}
			0 & 1\\
			1 & 0
		\end{array}\right]
		\left[\begin{array}{c}
			a\\
			b
		\end{array}\right] =
		\left[\begin{array}{c}
			a\\
			b
		\end{array}\right] \Rightarrow v_1 =
		\left[\begin{array}{c}
			1\\
			1
		\end{array}\right]
		\nonumber
	\end{equation}
	\begin{equation}
		\left[\begin{array}{cc}
			0 & 1\\
			1 & 0
		\end{array}\right]
		\left[\begin{array}{c}
			a\\
			b
		\end{array}\right] =
		\left[\begin{array}{c}
			-a\\
			-b
		\end{array}\right] \Rightarrow v_{-1} =
		\left[\begin{array}{c}
			1\\
			-1
		\end{array}\right]
		\nonumber
	\end{equation}
	Caso a seja positivo, pode-se concluir tamb\'em para $(\pm\frac{1}{\sqrt{a}}, 0)$ a partir
	da equa\c{c}\~ao \ref{eq:characteristic} que os autovalores para os dois s\~ao
	$\pm i\sqrt{2}$. Isso significa que os dois equil\'ibrios n\~ao s\~ao hiperb\'olicos
	e n\~ao \'e poss\'ivel determinar conclus\~oes a respeito de suas estabilidades somente
	com a jacobiana.

	Utilizando o mesmo c\'odigo do exerc\'icio 7.1, adaptando e modificando os valores iniciais
	para esta fun\c{c}\~ao, p\^ode-se construir diversos planos de fases para esta fam\'ilia
	de EDO's:
	\begin{figure}[H]
		\centering
		\subfigure[a = -1.]{
			\includegraphics[width=5.5cm]{phase_planes/a_minus.jpg}
		}
		\subfigure[a = 1.]{
			\includegraphics[width=5.5cm]{phase_planes/a_one.jpg}
		}
		\caption{Planos de fase para a EDO em pauta.}
	\end{figure}
	Como esperado, para a negativo, (0, 0), o \'unico ponto de equil\'ibrio, atua como ponto
	inst\'avel, gerando um plano de fase bem semelhante a de um mapa linear de equil\'ibrio
	correspondente. As curvas quando tomadas em mais tempo divergem para o infinito.

	Para a igual a 1, pode-se perceber que curvas peri\'odicas se formam ao redor dos pontos
	$(\pm\frac{1}{\sqrt{a}}, 0)$, o que sugere que eles sejam pontos est\'aveis, sem serem
	assintoticamente. O comportamento do ponto (0, 0) ainda \'e congruente com o previsto
	pela teoria, inst\'avel.
	
	Tamb\'em foi testado com valores menores que 1:
	\begin{figure}[H]
		\centering
		\subfigure[a = 0,5.]{
			\includegraphics[width=5.5cm]{phase_planes/a_half.jpg}
		}
		\subfigure[a = 0,25.]{
			\includegraphics[width=5.5cm]{phase_planes/a_quarter.jpg}
		}
		\caption{Planos de fase para a EDO em pauta.}
	\end{figure}
	Percebe-se que os pontos $(\pm\frac{1}{\sqrt{a}}, 0)$ se afastam da origem, pois o valor
	de a diminui, mas seus comportamentos continuam os mesmos, com curvas peri\'odicas
	ao seu redor. Estes mapas se equivalem ao a igual a 1 por\'em com uma escala cada vez
	maior.

	Por \'ultimo, testou-se com valores bem maiores que 1.
	\begin{figure}[H]
		\centering
		\subfigure[a = 10.]{
			\includegraphics[width=5.5cm]{phase_planes/a_ten.jpg}
		}
		\subfigure[a = 100.]{
			\includegraphics[width=5.5cm]{phase_planes/a_hundred.jpg}
		}
		\caption{Planos de fase para a EDO em pauta.}
	\end{figure}
	Claro que se esperava que os dois pontos $(\pm\frac{1}{\sqrt{a}}, 0)$ come\c{c}assem a
	se juntar, e eles se aglomeram na origem. Isso acaba gerando o resultado curioso em que
	os tr\^es pontos parecem o mesmo ponto e dessa forma todos est\~ao gerando curvas peri\'odicas.
	Assume-se que se houvesse uma an\'alise do que ocorre mais de perto, a diferen\c{c}a entre os
	tr\^es pontos se tornaria vis\'ivel, por\'em os caminhos nessa escala menor demoram muito tempo
	para se formarem, o que impede de realizar esse tipo de an\'alise computacionalmente.
	\section{\textit{Computer Experiment} 8.1}
	Utilizando e adaptando novamente o mesmo c\'odigo do exerc\'icio 7.1:
	\begin{figure}[H]
		\centering
		\subfigure[Trajet\'orias internas.]{
			\includegraphics[width=5.5cm]{phase_planes/van_der_pol_in.jpg}
		}
		\subfigure[Trajet\'orias externas.]{
			\includegraphics[width=5.5cm]{phase_planes/van_der_pol_out.jpg}
		}
		\caption{Planos de fase para a equa\c{c}\~ao de \textit{Van der Pol}.}
	\end{figure}
	Percebe-se que esta forma aproximadamente quadril\'atera \'e de fato o conjunto \^omega
	limite para praticamente todos os pontos iniciais. Por\'em:
	\begin{equation}
		\left\{\begin{array}{c}
			\dot{x} = y\\
			\dot{y} = -x - y(x^2 - 1)
		\end{array}\right.
		\Rightarrow y = 0 \Rightarrow x = 0
		\nonumber
	\end{equation}
	(0, 0) \'e ponto de equil\'ibrio da equa\c{c}\~ao. Ela obviamente \'e inst\'avel pela figura,
	por\'em uma \'orbita que comece neste ponto ter\'a como conjunto \^omega limite o ponto
	(0, 0), sendo este o outro conjunto deste tipo para a equa\c{c}\~ao.
\end{document}
