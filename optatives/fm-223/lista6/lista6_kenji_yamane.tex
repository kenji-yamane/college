\documentclass{article}[twocolumn]
\usepackage[pdftex]{graphicx}
\usepackage[utf8]{inputenc}
\usepackage[brazil]{babel}
\usepackage{subfigure}
\usepackage{mathtools}
\usepackage{amsmath}
\usepackage{amssymb}
\usepackage{float}
\usepackage{tikz}

\title{Lista 6}
\author{Kenji Yamane}

\begin{document}
	\maketitle
	\section{O conjunto dos inteiros de 1 a 100}
	Sabe-se que a \textit{box dimension} de um conjunto S \'e dado por:
	\begin{equation}
		boxdim(S) = \lim_{\epsilon \rightarrow 0}\frac{ln(N(\epsilon))}{ln(\frac{1}{\epsilon})}
		\label{eq:boxdim}
	\end{equation}
	No caso do conjunto dos n\'umeros inteiros de 1 a 100, o valor de $N(\epsilon)$ para
	$\epsilon$ menor que 1 sempre ser\'a 100, pois, toda caixa, por ter tamanho menor que 1,
	poder\'a cobrir um, e somente um dos inteiros. Ademais, n\~ao importa o qu\~ao pequeno seja
	$\epsilon$, sempre haver\'a 100 caixas que cobrem os 100 n\'umeros, as quais seriam:
	\begin{equation}
		c_{i} = \{x \in \mathbb{R} \; | \; i - \epsilon < x < i + \epsilon\}, \;
		i \in \{1, 2, 3, ..., 100\}
		\nonumber
	\end{equation}
	De tal forma que a caixa $c_{i}$ cobre o inteiro i.

	Portanto, a equa\c{c}\~ao \ref{eq:boxdim} se simplifica a:
	\begin{equation}
		boxdim(\{1, 2, 3, ..., 100\}) =
		\lim_{\epsilon \rightarrow 0}\frac{ln(100)}{ln(\frac{1}{\epsilon})} =
		\lim_{\eta \rightarrow \infty}\frac{ln(100)}{\eta} = 0
		\nonumber
	\end{equation}
	\section{O conjunto dos n\'umeros racionais de 0 a 1}
	J\'a no caso dos racionais entre 0 e 1, n\~ao importa o qu\~ao pequeno seja $\epsilon$,
	todas as $\frac{1}{\epsilon}$ caixas estar\~ao cobrindo algum n\'umero racional.

	Antes de se realizar a prova em si, provar-se-\'a o seguinte lema: seja um n\'umero
	irracional positivo $y \in \mathbb{I}$. Ent\~ao, existe um x positivo $\in \mathbb{Q}$ tal que
	$x < y$.

	Prova do lema:
	\begin{equation}
		\exists z = 10^{i}, \, i \in \mathbb{N} \; | \; zy > 1
		\Rightarrow x = \frac{1}{z} < y
		\nonumber
	\end{equation}
	Segue-se ent\~ao, na demonstra\c{c}\~ao requisitada:
	
	No caso de $\epsilon$ ser racional, considerando a caixa $c_{i}$, i arbitr\'ario:
	\begin{equation}
		c_{i} = \{x \in \mathbb{R} \; | \; (i - 1)\epsilon < x < i\epsilon\} \Rightarrow
		\nonumber
	\end{equation}
	\begin{equation}
		\Rightarrow y = \frac{(i - 1)\epsilon + i\epsilon}{2} = \frac{(2i - 1)\epsilon}{2}
		\Rightarrow y \in c_{i} \; e \; y \in \mathbb{Q}
		\nonumber
	\end{equation}
	No caso de $\epsilon$ ser irracional, pode-se, pelo lema provado anteriormente,
	escolher um $\epsilon' < \epsilon$, de tal forma que, para a mesma caixa $c_{i}$,
	pode-se escolher uma caixa $c_{j}'$ associada ao $\epsilon'$ tal que:
	\begin{equation}
		c_{j}' \subset c_{i}
		\label{eq:contains}
	\end{equation}
	A caixa $c_{j}'$ est\'a associada ao $\epsilon' \in \mathbb{Q}$. Portanto, como
	provado anteriormente:
	\begin{equation}
		\exists y \in \mathbb{Q} \; | \; y \in c_{j}' \xRightarrow{\ref{eq:contains}}
		y \in c_{i}
		\nonumber
	\end{equation}
	Portanto todas as $\frac{1}{\epsilon}$ caixas entre 0 e 1 cont\'em algum n\'umero
	racional para qualquer $\epsilon > 0$. Logo $N(\epsilon) = \frac{1}{\epsilon}$.
	
	Assim:
	\begin{equation}
		boxdim(\mathbb{Q}\cap[0, 1]) =
		\lim_{\epsilon \rightarrow 0}\frac{ln(N(\epsilon))}{ln(\frac{1}{\epsilon})} =
		\lim_{\epsilon \rightarrow 0}\frac{ln(\frac{1}{\epsilon})}{ln(\frac{1}{\epsilon})} =
		\lim_{\epsilon \rightarrow 0}1 = 1
		\nonumber
	\end{equation}
	O conjunto dos n\'umeros racionais \'e um conjunto cont\'avel, como provado na lista
	anterior. Al\'em disso, ele \'e de certa forma discreto, tendo n\'umeros irracionais
	entre seus elementos. Entretanto sua \textit{box counting dimension} resultou em 1.
	Essa ambival\^encia sugere comportamento fractal.
	\section{\textit{Box Counting Dimension}}
	Calculou-se a \textit{boxdim} de 10 milh\~oes de pontos  de uma \'orbita do
	seguinte mapa de H\'enon:
	\begin{equation}
		f(x, y) = (1.4 - x^2 + 0.3y, x)
		\nonumber
	\end{equation}
	Utilizou-se o algoritmo usual para o c\'alculo desse tipo de dimens\~ao. A seguir se
	tem, por exemplo, as caixas contadas para dois tamanhos $\epsilon$ diferentes.
	\begin{figure}[H]
		\subfigure[Figura obtida para $\epsilon$ igual a 0.125]{
			\includegraphics[width=5.5cm]{8bit.jpg}
		}
		\subfigure[Figura obtida para $\epsilon$ igual a 0.03125]{
			\includegraphics[width=5.5cm]{32bit.jpg}
		}
	\end{figure}
	Realizou-se a contagem de caixas para $\epsilon \in \{\frac{1}{8}, \frac{1}{16},
	\frac{1}{32}, \frac{1}{64}, \frac{1}{128}, \frac{1}{256}\}$ em \textit{C++} com o seguinte
	c\'odigo.
	\begin{verbatim}
#include <iostream>
#include <fstream>

#include <cmath>
#include <cstring>

#define ORBIT_SIZE 10000000


void henon_map(double arr[]) {
    double x = arr[0], y = arr[1];
    
    arr[0] = 1.4 - x*x + 0.3*y;
    arr[1] = x;
}

int main() {
    int marked = 0;
    double orbit[2] = {0, 0};
    bool grid[1024][1024] = {};
    std::ofstream o("boxes.txt", std::ofstream::out);

    for (int res = 8; res <= 256; res *= 2) {
        for (int i = 0; i < ORBIT_SIZE; i++) {
            if (i > 1000) {
                int i = floor(res*orbit[0] + 2*res);
                int j = floor(res*orbit[1] + 2*res);
                grid[i][j] = true;
                henon_map(orbit);
            }
        }

        for (int i = 0; i < 1024; i++) {
            for (int j = 0; j < 1024; j++) {
                if (grid[i][j]) {
                    marked++;
                }
            }
        }

        o << log2(res) << ",";
        o << log2(marked) << std::endl;
        memset(grid, sizeof(grid), false);
        marked = 0;
    }

    o.close();

    return 0;
}

	\end{verbatim}
	E em seguida, realizou-se a regress\~ao linear, e os gr\'aficos em \textit{Python},
	com o seguinte c\'odigo realizando a regress\~ao linear e seu gr\'afico.
	\begin{verbatim}
import csv
import numpy as np
from scipy import stats
import matplotlib.pyplot as plt

horizontal = []
vertical = []

with open('boxes.txt') as csvfile:
    spamreader = csv.reader(csvfile, delimiter=',')
    for row in spamreader:
        horizontal.append(float(row[0]))
        vertical.append(float(row[1]))

x = np.array(horizontal)
y = np.array(vertical)
dim, C, r_value, p_value, err = stats.linregress(x, y)
print('dimensao:', dim, err)
print('C:', C)
print('r2:', r_value**2)
print('p:', p_value)
plt.plot(horizontal, vertical, 'r.')
plt.plot(np.linspace(3, 8), dim*np.linspace(3, 8) + C)
plt.savefig('boxdim.jpg')

	\end{verbatim}
	Obtendo-se assim o seguinte gr\'afico de $log_{2}N(\epsilon)$ por
	$log_{2}\frac{1}{\epsilon}$.
	\begin{figure}[H]
		\centering
		\includegraphics[width=6.5cm]{boxdim.jpg}
		\caption{Regress\~ao obtida para as contagens de caixas.}
	\end{figure}
	Obteve-se um valor de 0.99 para o coeficiente de determina\c{c}\~ao, o que indica
	coer\^encia no modelo linear, e um valor de 1,38 para a dimens\~ao fractal do
	conjunto de H\'enon, bem diferente da obtida pelo livro. Isto pode ser advindo das
	diferentes ferramentas utilizadas, cujas diferen\c{c}as podem impactar um resultado
	caso este seja muito sens\'ivel.
	\section{\textit{Correlation dimension}}
	Utilizou-se da mesma forma, \textit{C++} para realizar a contagem, com uma \'orbita de
	tamanho 1000 e $r \in \{\frac{1}{4}, \frac{1}{8}, \frac{1}{16}, \frac{1}{32}, \frac{1}{64},
	\frac{1}{128}, \frac{1}{256}\}$, c\'odigo mostrado
	a seguir.
	\begin{verbatim}
#include <iostream>
#include <fstream>
#include <utility>

#include <cmath>
#include <cstring>

#define ORBIT_SIZE 1000


void henon_map(double arr[]) {
    double x = arr[0], y = arr[1];
    
    arr[0] = 1.4 - x*x + 0.3*y;
    arr[1] = x;
}

int main() {
    double arr[2] = {0, 0};
    double dist[1000000], dx, dy;
    int len = 0;
    std::pair<double, double> orbit[ORBIT_SIZE + 1];
    std::ofstream o("corr.txt", std::ofstream::out);
    	
    for (int i = 0; i < 2*ORBIT_SIZE; i++) {
        if (i >= ORBIT_SIZE) {
            orbit[i - ORBIT_SIZE].first = arr[0];
            orbit[i - ORBIT_SIZE].second = arr[1];
        }
        henon_map(arr);
    }
    	
    for (int i = 0; i < ORBIT_SIZE; i++) {
        for (int j = i + 1; j < ORBIT_SIZE; j++) {
            dx = orbit[i].first - orbit[j].first;
            dy = orbit[i].second - orbit[j].second;
            dist[len++] = sqrt(dx*dx + dy*dy);
        }
    }
    	
    double C = 0;
    double choose2 = ORBIT_SIZE*(ORBIT_SIZE - 1)/2;
    for (int res = 4; res <= 256; res *= 2) {
        for (int i = 0; i < len; i++) {
            if (dist[i] < 1.0/res) {
                C += 1;
            }
        }
        o << log2(1.0/res) << ",";
        o << log2(C/choose2) << std::endl;
        C = 0;
    }
    	
    o.close();
    
    return 0;
}

	\end{verbatim}
	Utilizando-se novamente da mesma forma que no \textit{box counting dimension} \textit{Python}
	para tratar os dados, com o seguinte c\'odigo.
	\begin{verbatim}
import csv
import numpy as np
from scipy import stats
import matplotlib.pyplot as plt

horizontal = []
vertical = []

with open('corr.txt') as csvfile:
    spamreader = csv.reader(csvfile, delimiter=',')
    for row in spamreader:
        horizontal.append(float(row[0]))
        vertical.append(float(row[1]))

x = np.array(horizontal)
y = np.array(vertical)
dim, C, r_value, p_value, err = stats.linregress(x, y)
print('dimensao:', dim, err)
print('C:', C)
print('r2:', r_value**2)
print('p:', p_value)
plt.plot(horizontal, vertical, 'r.')
plt.plot(np.linspace(-8, -2), dim*np.linspace(-8, -2) + C)
plt.savefig('cordim.jpg')
	\end{verbatim}
	Com este c\'odigo, obteve-se a seguinte regress\~ao.
	\begin{figure}[H]
		\centering
		\includegraphics[width=6.5cm]{cordim.jpg}
		\caption{Regress\~ao obtida para a dimens\~ao de correla\c{c}\~ao.}
	\end{figure}
	Obteve-se com este m\'etodo um coeficiente de determina\c{c}\~ao de 0.999, novamente
	confirmando o modelo linear e uma dimens\~ao de 1.28, de novo levemente maior que a
	determinada no livro, refor\c{c}ando a no\c{c}\~ao de uma certa imprecis\~ao.
\end{document}
