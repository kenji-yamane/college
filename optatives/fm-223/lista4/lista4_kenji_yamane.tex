\documentclass{article}[twocolumn]
\usepackage[pdftex]{graphicx}
\usepackage[utf8]{inputenc}
\usepackage[brazil]{babel}
\usepackage{subfigure}
\usepackage{mathtools}
\usepackage{amsmath}
\usepackage{amssymb}
\usepackage{float}
\usepackage{tikz}

\title{Lista 4}
\author{Kenji Yamane}

\begin{document}
	\maketitle
	\newpage
	\section{Determina\c{c}\~ao anal\'itica do ponto de sela}
	Mapas de H\'enon s\~ao da forma:
	\begin{equation}
		f_{a, b}(x, y) = (a - x^{2} + by, x)
		\nonumber
	\end{equation}
	No caso da figura 2.21a do livro:
	\begin{equation}
		\label{eq:221a}
		f(x, y) = (1.28 - x^{2} - 0.3y, x)
	\end{equation}
	Se ($x^{*}$, $y^{*}$) \'e ponto fixo de \ref{eq:221a}:
	\begin{equation}
		f(x^{*}, y^{*}) = (x^{*}, y^{*}) \xRightarrow{\ref{eq:221a}}
		\left\{\begin{array}{l}
			x^{*} = 1.28 - x^{*2} - 0.3y^{*}\\
			y^{*} = x^{*}
		\end{array}\right. \Rightarrow
		\nonumber
	\end{equation}
	\begin{equation}
		\Rightarrow x^{*2} + 1.3x^{*} - 1.28 = 0
		\Rightarrow x^{*} = \frac{-1.3 \pm \sqrt{1.3^{2} + 5.12}}{2} \Rightarrow
		\nonumber
	\end{equation}
	\begin{equation}
		\Rightarrow (x^{*}, y^{*}) \in \{(0.65, 0.65), (-1.95, -1.95)\}
		\nonumber
	\end{equation}
	Calculando a jacobiana de f:
	\begin{equation}
		Df = \left[\begin{array}{cc}
			\frac{\partial f_{1}}{\partial x} & \frac{\partial f_{1}}{\partial y}\\
			\frac{\partial f_{2}}{\partial x} & \frac{\partial f_{2}}{\partial x}
		\end{array}\right]
		= \left[\begin{array}{cc}
			-2x & -0.3\\
			1 & 0
		\end{array}\right]
		\nonumber
	\end{equation}
	Portanto:
	\begin{equation}
		\left\{\begin{array}{l}
			det(Df(0.65, 0.65) - \lambda I) = \lambda^{2} + 1.3\lambda + 0.3 = 0\\
			det(Df(-1.95, -1.95) - \lambda I) = \lambda^{2} - 3.9\lambda + 0.3 = 0
		\end{array}\right. \Rightarrow
		\nonumber
	\end{equation}
	\begin{equation}
		\Rightarrow
		\left\{\begin{array}{l}
			\lambda = \frac{-1.3 \pm \sqrt{1.3^{2} - 1.2}}{2}\\
			\lambda = \frac{3.9 \pm \sqrt{3.9^{2} - 1.2}}{2}
		\end{array}\right.
		\Rightarrow
		\left\{\begin{array}{l}
			\lambda \in \{-0.29, -1.01\} \Rightarrow (0.65, 0.65): sela\\
			\lambda \in \{3.83, 0.07\} \Rightarrow (-1.95, -1.95): sela
		\end{array}\right.
		\nonumber
	\end{equation}
	No caso da figura 2.21b do livro:
	\begin{equation}
		\label{eq:221b}
		f(x, y) = (1.4 - x^{2} - 0.3y, x)
	\end{equation}
	Se ($x^{*}$, $y^{*}$) \'e ponto fixo de \ref{eq:221b}:
	\begin{equation}
		f(x^{*}, y^{*}) = (x^{*}, y^{*}) \xRightarrow{\ref{eq:221b}}
		\left\{\begin{array}{l}
			x^{*} = 1.4 - x^{*2} - 0.3y^{*}\\
			y^{*} = x^{*}
		\end{array}\right. \Rightarrow
		\nonumber
	\end{equation}
	\begin{equation}
		\Rightarrow x^{*2} + 1.3x^{*} - 1.4 = 0
		\Rightarrow x^{*} = \frac{-1.3 \pm \sqrt{1.3^{2} + 5.6}}{2} \Rightarrow
		\nonumber
	\end{equation}
	\begin{equation}
		\Rightarrow (x^{*}, y^{*}) \in \{(0.7, 0.7), (-2, -2)\}
		\nonumber
	\end{equation}
	Calculando a jacobiana de f:
	\begin{equation}
		Df = \left[\begin{array}{cc}
			\frac{\partial f_{1}}{\partial x} & \frac{\partial f_{1}}{\partial y}\\
			\frac{\partial f_{2}}{\partial x} & \frac{\partial f_{2}}{\partial x}
		\end{array}\right]
		= \left[\begin{array}{cc}
			-2x & -0.3\\
			1 & 0
		\end{array}\right]
		\nonumber
	\end{equation}
	Portanto:
	\begin{equation}
		\left\{\begin{array}{l}
			det(Df(0.7, 0.7) - \lambda I) = \lambda^{2} + 1.4\lambda + 0.3 = 0\\
			det(Df(-2, -2) - \lambda I) = \lambda^{2} - 4\lambda + 0.3 = 0
		\end{array}\right. \Rightarrow
		\nonumber
	\end{equation}
	\begin{equation}
		\Rightarrow
		\left\{\begin{array}{l}
			\lambda = \frac{-1.4 \pm \sqrt{1.4^{2} - 1.2}}{2}\\
			\lambda = \frac{4 \pm \sqrt{4^{2} - 1.2}}{2}
		\end{array}\right.
		\Rightarrow
		\left\{\begin{array}{l}
			\lambda \in \{-0.26, -1.14\} \Rightarrow (0.7, 0.7): sela\\
			\lambda \in \{3.92, 0.08\} \Rightarrow (-2, -2): sela
		\end{array}\right.
		\nonumber
	\end{equation}
	\section{Determina\c{c}\~ao numer\'ica do ponto de sela}
	Criou-se a seguinte fun\c{c}\~ao, baseada no algoritmo apresentado nos slides:
	\begin{verbatim}
def inverse_jacobian_step(x, a):
    """
        Determines dx in the Newton-Raphson method
        :param x: two dimensional input
        :type x: numpy array
    """
    
    jacobian = np.array([[-2*x[0] - 1, -0.3], [1, -1]])
    applied = np.array([a - x[0]**2 - 0.3*x[1] - x[0], x[0] - x[1]])
    return np.array(np.linalg.solve(jacobian, applied))
	\end{verbatim}
	A partir dela se determina o dx. Utilizando como condi\c{c}\~ao de parada
	a soma dos quadrados de dx ser menor do que $10^{-8}$, obteve-se diferentes pontos
	de converg\^encia, dependendo do chute inicial e da escolha do par\^ametro a:
	\begin{table}[H]
		\centering
		\begin{tabular}{ccc}
			\hline
			a & Chute & Ponto obtido\\
			\hline
			1.28 & (1, 1) & (0.65479884, 0.65479884)\\
			1.28 & (-3, -3) & (-1.95479884, -1.95479884)\\
			1.4 & (1, 1) & (0.7, 0.7)\\
			1.4 & (-3, -3) & (-2, -2)\\
			\hline
		\end{tabular}
		\caption{Pontos fixos obtidos pelo algoritmo}
		\label{tab:newton-raphson}
	\end{table}
	O que \'e bastante coerente com os resultados obtidos analiticamente, com uma pequena
	diverg\^encia com rela\c{c}\~ao aos pontos fixos da primeira fun\c{c}\~ao. Por\'em,
	essas diferen\c{c}as s\~ao justificadas pelas aproxima\c{c}\~oes feitas na dedu\c{c}\~ao
	anal\'itica, sendo portanto os resultados obtidos numericamente mais pr\'oximo do
	verdadeiro, provavelmente.
	\section{Determina\c{c}\~ao de variedades est\'avel e inst\'avel}
	Para se determinar as variedades de um ponto de sela de um mapa, deve-se primeiramente
	encontrar os autovetores. Para um mapa de H\'enon qualquer:
	\begin{equation}
		Df(x, y) = \left[\begin{array}{cc}
			2x & -0.3\\
			1 & 0
		\end{array}\right]
	\end{equation}
	Sejam (a, b) um autovetor correspondente ao autovalor $\lambda$ para um dado ponto. Assim:
	\begin{equation}
		\left[\begin{array}{cc}
			2x^{*} & -0.3\\
			1 & 0
		\end{array}\right]
		\left[\begin{array}{c}
			a \\ b
		\end{array}\right]
		= \left[\begin{array}{cc}
			\lambda a \\ \lambda b
		\end{array}\right]
		\xRightarrow{det(Df) = 0} a = \lambda b
		\nonumber
	\end{equation}
	Dessa forma, como j\'a se tem o autovalor da se\c{c}\~ao anterior, tem-se os autovetores
	necess\'arios para construir as variedades com o c\'odigo de \textit{kostelich}.

	Escolhendo valores apropriados para os par\^ametros dispon\'iveis e escolhendo-se a
	orienta\c{c}\~ao correta dos autovetores, p\^ode-se construir as variedades para os
	dois mapas de Hen\'on, mostradas em \ref{fig:manifolds}.
	\begin{figure}[H]
		\centering
		\begin{subfigure}
			\centering
			\includegraphics[width=10cm]{221a.jpg}
		\end{subfigure}
		\begin{subfigure}
			\centering
			\includegraphics[width=10cm]{221b.jpg}
		\end{subfigure}
		\caption{Variedades econtradas (a = 1.28 e a = 1.4).}
		\label{fig:manifolds}
	\end{figure}
	\section{Expoente de \textit{Lyapunov}}
	Escerveu-se o seguinte c\'odigo para gerar a figura requisitada.
	\begin{verbatim}
	import matplotlib.pyplot as plt
	import numpy as np
	
	def logistic_function(a, x):
		return a*x*(1 - x)
	
	def logistic_derivative(a, x):
		return a*(1 - 2*x)
	
	a_space = np.linspace(2, 4, 60000)
	lyapunov_a_space = []
	lyapunov_exponent_space = []
	
	for a in a_space:
		orbit = np.random.uniform(0, 1)
		for i in range(100):
			orbit = logistic_function(a, orbit)
			
		lyapunov_exponent = 0
		for i in range(100):
			orbit = logistic_function(a, orbit)
			if logistic_derivative(a, orbit) != 0:
				lyapunov_exponent += np.log(np.abs(logistic_derivative(a, orbit)))
				lyapunov_exponent /= 100
		if lyapunov_exponent > -1:
			lyapunov_exponent_space.append(lyapunov_exponent)
		lyapunov_a_space.append(a)
		
	lyapunov_exponent_space = np.array(lyapunov_exponent_space)
	plt.plot(lyapunov_a_space, lyapunov_exponent_space, ',')
	plt.plot(a_space, np.zeros(len(a_space)))
	plt.savefig('lyapunov.jpg')

	\end{verbatim}
	Obtendo-se assim a figura \ref{fig:lyapunov}.
	\begin{figure}[H]
		\centering
		\includegraphics[width=10cm]{lyapunov.jpg}
		\caption{Expoente de \textit{Lyapunov} para o mapa log\'istico}
		\label{fig:lyapunov}
	\end{figure}
\end{document}
