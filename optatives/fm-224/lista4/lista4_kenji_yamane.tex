\documentclass{article}[twocolumn]
\usepackage[pdftex]{graphicx}
\usepackage[utf8]{inputenc}
\usepackage[brazil]{babel}
\usepackage{subfigure}
\usepackage{mathtools}
\usepackage{amsmath}
\usepackage{amssymb}
\usepackage{float}
\usepackage{tikz}

\title{Lista 4}
\author{Kenji Yamane}

\begin{document}
	\maketitle
	\section{Ser\'ie de atra\c{c}\~ao com ru\'ido}
	Para realizar o integrador de Runge-Kutta de quarta ordem e passo fixo no tempo para as
	equa\c{c}\~oes das ondas Alfv\'en, reaproveitou-se o c\'odigo do exame de FM-223, adicionando
	a ele os ru\'idos que tornam a \'orbita estoc\'astica. Em seguida, criou-se o c\'odigo que
	gera a \'orbita a ser mostrada na s\'erie temporal:
	\begin{verbatim}
#include <stdio.h>
#include <stdlib.h>
#include <stdbool.h>
#include <time.h>

#include "solve.h"

enum PERIOD {
    ONE = 0,
    THREE,
    SIX
};

#define DISCARDED 1000
#define CONV 1000
#define ORBIT_LEN 20000
#define EPSILON 1e-3

#define FRAMES_PER_PERIOD 30
#define STEP 2*M_PI/FRAMES_PER_PERIOD

void stochastic_stroboscopic_map(double orbit_point[3]) {
    for (int i = 0; i < FRAMES_PER_PERIOD; i++) {
        rk_fourth_order(orbit_point, STEP, true);
    }
}

void stroboscopic_map(double orbit_point[3]) {
    for (int i = 0; i < FRAMES_PER_PERIOD; i++) {
        rk_fourth_order(orbit_point, STEP, false);
    }
}

enum PERIOD which_attractor(double orbit_point[3]) {
    double previous_bz;
    double predict_point[3] = {orbit_point[0], orbit_point[1], orbit_point[2]};

    for (int i = 0; i < CONV; i++) {
        stroboscopic_map(predict_point);
    }

    previous_bz = predict_point[1];
    stroboscopic_map(predict_point);
    if (fabs(predict_point[1] - previous_bz) < EPSILON) return ONE;

    previous_bz = predict_point[1];
    for (int i = 0; i < 3; i++) stroboscopic_map(predict_point);
    if (fabs(predict_point[1] - previous_bz) < EPSILON) return THREE;

    return SIX;
}

int main() {
    double pTime = 0;
    srand(time(NULL));
    
    FILE *s = fopen("diagram.txt", "w");

    double orbit_point[3] = {-0.9, 1, 0};
    for (int i = 0; i < DISCARDED; i++) {
        stochastic_stroboscopic_map(orbit_point);
        pTime += 2*M_PI;
    }
    for (int i = 0; i < ORBIT_LEN; i++) {
        stochastic_stroboscopic_map(orbit_point);
        pTime += 2*M_PI;
        fprintf(s, "%lf, %lf, %d\n", pTime, orbit_point[1], which_attractor(orbit_point));
    }

    fclose(s);

    return 0;
}
	\end{verbatim}
	Com os dados gerados por este arquivo, p\^ode-se plotar com \textit{Python} a figura
	requisitada:
	\begin{figure}[H]
		\centering
		\includegraphics[width=8cm]{temporal_diagram.png}
		\caption{S\'erie temporal da \'orbita estoc\'astica.}
	\end{figure}
	Utilizou-se os mesmos marcadores dos \textit{slides}  para identificar a bacia \`a
	qual o ponto da \'orbita pertence. Observa-se que n\~ao est\'a igual, o que era bem esperado
	dado que se est\'a lidando com um mapa estoc\'astico. Entretanto, observa-se bem a maior
	concentra\c{c}\~ao na bacia atratora vinculada a \'orbita de per\'iodo unit\'ario, com
	entretanto uma certa intermit\^encia com as outras duas bacias, o que revela a complexidade
	presente. A imagem mostra portanto, o que era previsto.
\end{document}
