\documentclass{article}[twocolumn]
\usepackage[pdftex]{graphicx}
\usepackage[utf8]{inputenc}
\usepackage[brazil]{babel}
\usepackage{subfigure}
\usepackage{mathtools}
\usepackage{amsmath}
\usepackage{amssymb}
\usepackage{float}
\usepackage{tikz}

\usepackage[a4paper,top=2.5cm,bottom=2.5cm,left=2cm,right=2cm,marginparwidth=1.5cm]{geometry}

\title{Lista 8}
\author{Kenji Yamane}

\begin{document}
	\maketitle
	\section{Equa\c{c}\~ao \textit{Benjamin-Bona-Mahony}}
	Substituindo-se $u(x, t) = \sum_{k = -\infty}^{\infty} b_k(t)e^{ikx}$ na equa\c{c}\~ao
	de BBM, mostrada a seguir:
	\begin{equation}
		\frac{\partial u}{\partial t} + a\frac{\partial^3 u}{\partial x^2\partial t} + c\frac{\partial u}{\partial x}
		+ fu\frac{\partial u}{\partial x} = -\gamma u - \epsilon sen(x - \Omega t)
		\nonumber
	\end{equation}
	Tem-se:
	\begin{equation}
		\frac{\partial}{\partial t}\left(\sum_{k = -\infty}^{\infty} b_k(t)e^{ikx}\right) +
		a\frac{\partial^3}{\partial x^2\partial t}\left(\sum_{k = -\infty}^{\infty} b_k(t)e^{ikx}\right) +
		c\frac{\partial}{\partial x}\left(\sum_{k = -\infty}^{\infty} b_k(t)e^{ikx}\right) +
		\nonumber
	\end{equation}
	\begin{equation}
		f\left(\sum_{k = -\infty}^{\infty} b_k(t)e^{ikx}\right)\frac{\partial}{\partial x}\left(\sum_{k = -\infty}^{\infty} b_k(t)e^{ikx}\right)
		= -\gamma \sum_{k = -\infty}^{\infty} b_k(t)e^{ikx} - \epsilon sen(x - \Omega t) \Rightarrow
		\nonumber
	\end{equation}
	\begin{equation}
		\Rightarrow \sum_{k = -\infty}^{\infty} \dot{b}_k(t)e^{ikx} -
		a\sum_{k = -\infty}^{\infty} k^2\dot{b}_k(t)e^{ikx} +
		c\sum_{k = -\infty}^{\infty} ikb_k(t)e^{ikx} +
		f\sum_{k = -\infty}^{\infty}\sum_{m = -\infty}^{\infty} imb_k(t)b_m(t)e^{i(k + m)x}
		\nonumber
	\end{equation}
	\begin{equation}
		= -\gamma \sum_{k = -\infty}^{\infty} b_k(t)e^{ikx} - \epsilon sen(x - \Omega t)
		\nonumber
	\end{equation}
	Multiplicando por $e^{-ilx}$ e integrando no dom\'inio do espa\c{c}o, de 0 a $2\pi$, tem-se:
	\begin{equation}
		2\pi \dot{b}_k(t) - 2a\pi k^2\dot{b}_k(t) + 2c\pi ikb_k(t) + 2\pi f\sum_{m = -\infty}^{\infty} imb_m(t)b_{k - m}(t)
		\nonumber
	\end{equation}
	\begin{equation}
		= -2\pi\gamma b_k(t) - \int_0^{2\pi}\epsilon e^{-ikx}sen(x - \Omega t) \Rightarrow
		\nonumber
	\end{equation}
	\begin{equation}
		\dot{b}_k(t) - ak^2\dot{b}_k(t) + cikb_k(t) + if\sum_{m = -\infty}^{\infty} mb_m(t)b_{k - m}(t)
		= -\gamma b_k(t) - \frac{\epsilon}{2\pi}\int_0^{2\pi}e^{-ikx}sen(x - \Omega t)dx
		\label{eq:BBMincomplete}
	\end{equation}
	Resolvendo a integral desta equa\c{c}\~ao:
	\begin{equation}
		\int_0^{2\pi}e^{-ikx}sen(x - \Omega t)dx = -\frac{1}{ik}e^{-ikx}sen(x - \Omega t)\big|_{x = 0}^{2\pi}
		+ \frac{1}{ik}\int_0^{2\pi}e^{-ikx}cos(x - \Omega t)dx \Rightarrow
		\nonumber
	\end{equation}
	\begin{equation}
		\Rightarrow \int_0^{2\pi}e^{-ikx}sen(x - \Omega t)dx = \frac{1}{ik}\left(-\frac{1}{ik}e^{-ikx}cos(x - \Omega t)\big|_{x = 0}^{2\pi}
		- \frac{1}{ik}\int_0^{2\pi}e^{-ikx}sen(x - \Omega t)dx\right) \Rightarrow
		\nonumber
	\end{equation}
	\begin{equation}
		\Rightarrow \int_0^{2\pi}e^{-ikx}sen(x - \Omega t)dx = \frac{1}{k^2}\int_0^{2\pi}e^{-ikx}sen(x - \Omega t)dx \Rightarrow
		\nonumber
	\end{equation}
	\begin{equation}
		\Rightarrow (k^2 - 1)\int_0^{2\pi}e^{-ikx}sen(x - \Omega t)dx = 0
		\nonumber
	\end{equation}
	Assim, para $k \neq 1$, a integral \'e nula.
	
	Para k = 1:
	\begin{equation}
		\int_0^{2\pi}e^{-ix}sen(x - \Omega t)dx = \int_0^{2\pi}[cos(x) - isen(x)]sen(x - \Omega t)dx =
		\nonumber
	\end{equation}
	\begin{equation}
		\int_0^{2\pi}cos(x)sen(x - \Omega t)dx - i\int_0^{2\pi}sen(x)sen(x - \Omega t)dx
		\nonumber
	\end{equation}
	Pelas f\'ormulas de Prostaf\'erese:
	\begin{equation}
		\int_0^{2\pi}e^{-ix}sen(x - \Omega t)dx = \frac{1}{2}\int_0^{2\pi}sen(2x - \Omega t) - sen(\Omega t)dx -
		\frac{i}{2}\int_0^{2\pi}cos(\Omega t) - cos(2x - \Omega t)dx \Rightarrow
		\nonumber
	\end{equation}
	\begin{equation}
		\Rightarrow \int_0^{2\pi}e^{-ix}sen(x - \Omega t)dx = -\pi sen(\Omega t) - i\pi cos(\Omega t)
		\nonumber
	\end{equation}
	Aplicando isto \`a equa\c{c}\~ao \ref{eq:BBMincomplete}:
	\begin{equation}
		\dot{b}_k(t) - ak^2\dot{b}_k(t) + cikb_k(t) + if\sum_{m = -\infty}^{\infty} mb_m(t)b_{k - m}(t)
		= -\gamma b_k(t) + \frac{\epsilon \delta_{1, k}}{2}\left[sen(\Omega t) + cos(\Omega t)\right] \Rightarrow
		\nonumber
	\end{equation}
	\begin{equation}
		\Rightarrow \dot{b}_k = \frac{1}{1 - ak^2}\left\{-(ick + \gamma)b_k - if\sum_{m = -\infty}^{\infty} mb_mb_{k - m}
		+ \frac{\epsilon \delta_{1, k}}{2}\left[sen(\Omega t) + cos(\Omega t)\right]\right\}
		\nonumber
	\end{equation}
	Como queria-se demonstrar.
	\section{Equa\c{c}\~ao \textit{Kuramoto-Sivashinsky}}
	Implementando-se o m\'etodo espectral nesta equa\c{c}\~ao como est\'a nos slides, e realizando
	a integra\c{c}\~ao das equa\c{c}\~oes diferenciais resultantes em \texttt{Julia}, obteve-se a seguinte
	figura.
	\begin{figure}[H]
		\centering
		\includegraphics[width=9cm]{../bifurcation_diagram.png}
	\end{figure}
	Onde se observa similaridade satisfat\'oria com a figura original.
\end{document}

