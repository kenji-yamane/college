\documentclass{article}[twocolumn]
\usepackage[pdftex]{graphicx}
\usepackage[utf8]{inputenc}
\usepackage[brazil]{babel}
\usepackage{subfigure}
\usepackage{mathtools}
\usepackage{amsmath}
\usepackage{amssymb}
\usepackage{float}
\usepackage{tikz}

\title{Lista 8}
\author{Kenji Yamane}

\begin{document}
	\maketitle
	\section{Quest\~ao 2}
	Para os quesitos desta quest\~ao, utilizou-se o seguinte c\'odigo:
	\begin{verbatim}
import numpy as np
import matplotlib.pyplot as plt

STEP = 0.001
NUM_ORBITS = 100
DISCARDED_LENGTH = 40000
ORBIT_LENGTH = 1000

horizontal = []
vertical = []

def edo_func(x, y, r):
	return np.array([r*x + y - x**3, -x + r*y + 2*y**3])

def runge_kutta(x, y, control_param):
	v1 = STEP*edo_func(x, y, control_param)
	v1prime = np.array([x, y]) + v1/2
	v2 = STEP*edo_func(v1prime[0], v1prime[1], control_param)
	n = np.array([x, y]) + v2
	
	return n[0], n[1]

def search_stables(control_param):
	for i in range(NUM_ORBITS):
		x = np.random.uniform(-0.1, 0.1)
		y = np.random.uniform(-0.1, 0.1)
		
		for j in range(DISCARDED_LENGTH):
			x, y = runge_kutta(x, y, control_param)
		
		for j in range(ORBIT_LENGTH):
			x, y = runge_kutta(x, y, control_param)
			
			horizontal.append(x)
			vertical.append(y)

search_stables(0.01)

plt.xlabel('x')
plt.ylabel('y')
plt.plot(horizontal, vertical, 'r,')
plt.savefig('after2.png')

	\end{verbatim}
	\subsection{a}
	Nesta quest\~ao, utilizando-se o valor de 0,1 para a vari\'avel de controle,
	partindo-se de 100 pontos amostrados aleatoriamente entre -1 e 1 para cada uma
	de suas coordenadas, observou-se o seguinte.
	\begin{figure}[H]
		\centering
		\includegraphics[width=6cm]{after1.png}
		\caption{Retrato de fase para r = 0,1.}
	\end{figure}
	Onde claramente se observa um ciclo est\'avel.
	Para um valor antes de 0 para a vari\'avel de controle, -0,1, teve-se mais
	dificuldade na an\'alise, portanto constru\'iram-se tr\^es retratos de fase,
	gradativamente aumentando-se o tamanho da \'orbita de descarte.
	\begin{figure}[H]
		\centering
		\subfigure{\includegraphics[width=5cm]{before1_short.png}}
		\subfigure{\includegraphics[width=5cm]{before1_med.png}}
		\subfigure{\includegraphics[width=5cm]{before1_long.png}}
		\caption{Retratos de fase para r = -0,1.}
	\end{figure}
	Apesar de n\~ao parecer, se analisar a escala utilizada em cada figura, percebe-se
	claramente que as \'orbitas est\~ao convergindo para o (0, 0).

	Dado que foi afirmado que \'e uma bifurca\c{c}\~ao \textit{Hopf}, com estas informa\c{c}\~oes,
	pode-se concluir que ela \'e supercr\'itica. Os retratos de fase para -0,1 foram
	constru\'idos com diversas tentativas de amostragem aleat\'orias at\'e m\'odulo de 10 e
	p\^ode-se confirmar que n\~ao h\'a outro ciclo est\'avel.
	
	\subsection{b}
	Utilizando-se 0,01 para r, e amostrando-se aleatoriamente x e y entre -0.1 e 0.1,
	tem-se o seguinte retrato de fase.
	\begin{figure}[H]
		\centering
		\includegraphics[width=6cm]{after2.png}
		\caption{Retrato de fase para r = 0,01.}
	\end{figure}
	Onde nota-se como n\~ao h\'a converg\^encia para (0, 0), tendo pontos t\~ao longe
	quanto 0,4. Portanto, (0, 0) \'e um ponto inst\'avel.

	Utilizando-se -0,01 para r:
	\begin{figure}[H]
		\centering
		\includegraphics[width=6cm]{before2.png}
		\caption{Retrato de fase para r = -0,01.}
	\end{figure}
	Verifica-se como h\'a converg\^encia para o (0, 0) e ainda uma dist\^ancia consider\'avel
	entre uma \'orbita aparentemente c\'iclica e a nuvem de converg\^encia. Ademais,
	aumentando-se o alcance da amostragem, pontos come\c{c}am a divergir. Portanto,
	\'e razo\'avel deduzir que haja um ciclo inst\'avel ao redor de um ponto est\'avel.

	Essas caracter\'isticas configuram uma bifurca\c{c}\~ao \textit{Hopf} subcr\'itica.
\end{document}
